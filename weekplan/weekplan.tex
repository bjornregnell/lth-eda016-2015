För tider och salar se schema i
\href{http://cs.lth.se/eda016/schema}{TimeEdit}.

\subsection{Läsperiod 1}\label{lasperiod-1}

\begin{longtable}[c]{@{}llllll@{}}
\toprule\addlinespace
Vecka & Datum & Föreläsning & Resurstid & Laboration & Kontroll
\\\addlinespace
\midrule\endhead
V1 & 31/8-6/9 &
\href{http://fileadmin.cs.lth.se/cs/Education/EDA016/lectures/f1.pdf}{F1}
F2 & Ö1 Hello & Lab1 Quiz &
\\\addlinespace
V2 & & F3 -- & Ö2 Paket & Lab2 Eclipse &
\\\addlinespace
V3 & & F4 -- & Ö3 Objekt & Lab3 Anv. Square &
\\\addlinespace
V4 & & F5 F6 & Ö4 Räkna & Lab4 Impl. Square &
\\\addlinespace
V5 & & F7 F8 & Ö5 Klass & Lab5 Gissa Tal &
\\\addlinespace
V6 & & F9 F10 & Ö6 Vektor & Lab6 Turtle &
\\\addlinespace
V7 & & F11 F12 & Ö7 Registr. & Lab7 Maze &
\\\addlinespace
V8 & & -- & -- & -- & KS*
\\\addlinespace
\bottomrule
\end{longtable}

KS = Kontrollskrivning; obligatorisk, diagnostisk, kamraträttad, kan ge
samarbetsbonus.

\subsection{Läsperiod 2}\label{lasperiod-2}

\begin{longtable}[c]{@{}llllll@{}}
\toprule\addlinespace
Vecka & Datum & Förel. & Övn & Lab & Kontroll
\\\addlinespace
\midrule\endhead
V1 & 2/11-8/11 &
\href{http://fileadmin.cs.lth.se/cs/Education/EDA016/lectures/f1.pdf}{F1}
F2 & Ö1-hello & Lab1-quiz &
\\\addlinespace
V2 & & F3 -- & Ö2-kodfiler & Lab2-eclips &
\\\addlinespace
V3 & & F4 -- & Ö3-klasser & Lab3-vektor &
\\\addlinespace
V4 & & F5 F6 & Ö4-klasser & Lab3-vektor &
\\\addlinespace
V5 & & F7 F8 & Ö5-klasser & Lab3-vektor &
\\\addlinespace
V6 & & F9 F10 & Ö6-klasser & Lab3-vektor &
\\\addlinespace
V7 & & F11 F12 & Ö7-klasser & Lab3-vektor &
\\\addlinespace
V8 & & -- & -- & -- & Tenta
\\\addlinespace
\bottomrule
\end{longtable}

Tenta = Skriftlig tentamen utan hjälpmedel, förutom
\href{http://fileadmin.cs.lth.se/cs/Education/EDA016/general/quickref-booklet.pdf}{snabbreferens}.
