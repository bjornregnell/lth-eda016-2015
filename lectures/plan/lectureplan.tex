\href{https://github.com/bjornregnell/lth-eda016-2015/tree/master/lectures/notes}{Föreläsningsbilder
på GitHub}.
\href{http://cs.lth.se/eda016/veckoplanering}{Veckoplanering}.
\href{http://cs.lth.se/eda016/schema}{TimeEdit}.

\begin{longtable}[c]{@{}lllll@{}}
\toprule\addlinespace
Vecka & Föreläsning & Tema & Innehåll & Ankboken
\\\addlinespace
\midrule\endhead
W01 (Lp1V1) & F1 F2 & Introduktion & Om kursen, programmeringens
grundprinciper, programmeringsparadigmer, editera-kompilera-exekvera,
datorns delar, virtuell maskin, värde, uttryck, variabel, typ,
tilldelning, utdata med System.out, indata med Scanner, alternativ, if,
else, true, false & 1, 3.1-3.3, 5.1-5.3, 6.1-6.2, 7.1, 7.3
\\\addlinespace
W02 & F3 & Kodstruktur & loop-strukturer: while-sats, for-sats,
algoritm: min/max, Integer.MIN\_VALUE, Integer.MAX\_VALUE, Paket,
import, filstruktur, jar, dokumentation, programlayout, JDK, konstanter
vs föränderlighet, objektorientering, klasser, objekt,
referensvariabler, referenstilldelning, anropa metoder, SimpleWindow &
2.1-2.6, 4, 5.4, 7.2, 7.5-7.6, 7.8-7.9
\\\addlinespace
W03 & F4 & Systemutveckling & Krav-design-test, specifikationer, använda
vs implementera, exempel: Square, attribut, synlighetsregler, private,
public, konstruktor, this, implementera metoder, funktioner vs
procedurer, void, parametrar, Eclipse IDE, öppen källkod, Stack
overflow, GitHub \& Bitbucket & 2.7-2.10, 3.3-3.12
\\\addlinespace
W04 & F5 F6 & Aritmetik, Logik, Datastrukturer & primitiva typer
max/min-värden, klassen Math, precisionsproblem, attribut, implicita
startvärden, typkonvertering, modulo-räkning, förk. tilld., summering,
logiska uttryck, De Morgans lagar, enkel datastruktur: post som samlar
olika element, delade objektreferenser, oföränderlighet, konstanter,
exempel: Person, Square Point & 3.1-3.9, 5, 6.1-6.4, 7.2, 7.5
\\\addlinespace
W05 & F7 F8 & Klasser, Strängar, StringBuilder, Slumptal & char,
escape-tecken, formatering med printf, standadklasser: Character,
String, StringBuilder, Random, PrintWriter, satser: switch, break,
do-while, exempel: Datakomprimering, Text, Die, Player, DiceGame, skriva
strängar till fil & 11, 7.9, 6.10, 7.7, 7.4, 7.12
\\\addlinespace
W06 & F9 F10 & Vektorer & vektorer: deklarera, indexera, initialisera,
vektorer med referensvariabler, exempel: Fibonacci, Polygon, algoritmer:
summering, min/max, linjärsökning, insättning utan/med utökning,
borttagning, registrering av värden och intervall & 8
\\\addlinespace
W07 & F11 F12 & Arv & superklass, subklass, extends, super, instanceof,
klassen Object, implementera equals, repetition baserat på önskemål &
9.1, 9.3, 9.7-9.9, 11.2, 12.6
\\\addlinespace
W09 (Lp2V1) & F13 & Matriser & & Kapitel 8.6-8.7
\\\addlinespace
W10 (Lp2V2) & F14 & Listor & ArrayList, typklasser, autoboxning & 12
\\\addlinespace
W11 (Lp2V3) & F15 F16 & Polymorfism & skyddsregler vid arv, protected,
abstrakta metoder, typregler vid arv, instanceof, definitiva metoder och
klasser, @override & 9.2, 9.4-9.6, 9.10
\\\addlinespace
W12 (Lp2V4) & F17 F18 & Algoritmer & binärsökning, urvalssortering,
bubbelsortering, insättningssortering, komplexitet & 7.7, 8
\\\addlinespace
W13 (Lp2V5) & F19 & Designexempel & att skriva stora program & 9, 10, 13
\\\addlinespace
W14 (Lp2V6) & F20 & Repetition, Extentor & Vad kommer på tentan? & 1-13,
A, B, C
\\\addlinespace
W15 (Lp2V7) & F21 & Utblick & framtidens systemutveckling, kommande
kurser Pfk m.fl., ämnen på begäran & (14-16)
\\\addlinespace
\bottomrule
\end{longtable}
