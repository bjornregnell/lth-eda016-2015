För översikt av kursmoment se
\href{http://cs.lth.se/eda016/veckoplanering}{veckoplanering}. För tider
och salar se schema i \href{http://cs.lth.se/eda016/schema}{TimeEdit}.

\begin{longtable}[c]{@{}lllll@{}}
\toprule\addlinespace
Vecka & Föreläsning & Tema & Innehåll & Ankboken
\\\addlinespace
\midrule\endhead
Lp1V1 & F1 F2 & Introduktion & Om kursen, programmeringens
grundprinciper, programmeringsparadigmer, editera-kompilera-exekvera,
datorns delar, virtuell maskin, värde, uttryck, variabel, typ,
tilldelning, utdata med System.out, indata med Scanner, alternativ, if,
else, true, false & Kapitel 1, 3.1-3.3, 5.1-5.3, 6.1-6.2, 7.1, 7.3,
\\\addlinespace
Lp1V2 & F3 & Kodstruktur & repetition, while, for, algoritm: min/max,
Integer.MIN\_VALUE, Integer.MAX\_VALUE, Paket, import, filstruktur, jar,
dokumentation, programlayout, JDK, konstanter vs föränderlighet,
objektorientering, klasser, objekt, referensvariabler,
referenstilldelning, anropa metoder, exempel: SimpleWindow & Kapitel
2.1-2.6, 4, 5.4, 7.2, 7.5-7.6, 7.8-7.9
\\\addlinespace
Lp1V3 & F4 & Systemutveckling & Krav-design-test, specifikationer,
använda vs implementera, exempel: Square, attribut, synlighetsregler,
private, public, konstruktor, this, implementera metoder, funktioner vs
procedurer, void, parametrar, Eclipse IDE, öppen källkod, Stack
overflow, Github & Kapitel 2.7-2.10, 3.4-3.12,
\\\addlinespace
Lp1V4 & F5 F6 & Aritmetik, Logik, Datastrukturer & exempel: Point &
Kapitel 6.3-6.9
\\\addlinespace
Lp1V5 & F7 F8 & Klasser, Strängar, Slumptal & exempel: Person, Die,
switch, char, String, do-while & Kapitel 11.1-11.3, 6.10-6.11, 7.4, 7.7
\\\addlinespace
Lp1V6 & F9 F10 & Vektorer, Registrering & & Kapitel 8
\\\addlinespace
Lp1V7 & F11 F12 & Likhet, Synlighet, StringBuilder & & Kapitel ?-?
\\\addlinespace
Lp2V1 & F13 & Matriser & & Kapitel 8.6-8.7
\\\addlinespace
Lp2V2 & F14 & Listor & ArrayList, typklasser, autoboxning & Kapitel 12
\\\addlinespace
Lp2V3 & F15 F16 & Arv & & Kapitel 9
\\\addlinespace
Lp2V4 & F17 F18 & Algoritmer & linjärsökning, binärsökning,
urvalssortering, bubbelsortering, insättningssortering, komplexitet &
Kapitel 7.7, 8
\\\addlinespace
Lp2V5 & F19 & Designexempel & att skriva stora program, om
fördjupningskursen & Kapitel 10, 13, (14-16)
\\\addlinespace
Lp2V6 & F20 & Extentor, Repetition & &
\\\addlinespace
Lp2V7 & F21 & Utblick, Om tentan & framtidens systemutveckling, kommande
kurser, ämnen på begäran &
\\\addlinespace
\bottomrule
\end{longtable}
