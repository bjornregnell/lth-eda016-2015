\documentclass{lecturenotes}

\renewcommand{\vecka}{7}
\newcommand{\tema}{Arv}

\setbeamertemplate{footline}[frame number]
\title[Föreläsningsanteckningar EDA016, 2015]{EDA016 Programmeringsteknik för D}
\subtitle{Läsvecka \vecka: \tema}
\author{Björn Regnell}
\institute{Datavetenskap, LTH}
\date{Lp1-2, HT 2015}
 
\begin{document}

\frame{\titlepage}
\setnextsection{\vecka}
\section[Vecka \vecka: \tema]{\tema}
\frame{\tableofcontents}

%%%%%%%%%%%%%%%%%%%%%%%%%%%%%%%%%%%%%%

\subsection{Att göra denna vecka}
\begin{Slide}{Att göra i Vecka \vecka: Förstå \href{https://sv.wikipedia.org/wiki/Arv_\%28programmering\%29}{arv}. Repetera inför kontrollskrivning.}
\begin{enumerate}
\item Läs följande kapitel i kursboken: ??? \\  
Begrepp: registrering, equals, arv.
\item Gör övning 7: registrering
\item Träffas i samarbetsgrupper och hjälp varandra 
\item Gör Lab 6: Implementera Turtle och ColorTurtle.
\end{enumerate}
\end{Slide}

\subsection{Arv}
% Ide till övning: medelvärde (heltal i vektor) med tillämpning på bonusuträkning. 
%% Quest: Hur ska vi göra med decimaler? (Reellt beslut i kursen krävs då detta ej är def. i kursprog)


\end{document}