\documentclass{lecturenotes}

\renewcommand{\vecka}{7}
\newcommand{\tema}{Arv}

\setbeamertemplate{footline}[frame number]
\title[Föreläsningsanteckningar EDA016, 2015]{EDA016 Programmeringsteknik för D}
\subtitle{Läsvecka \vecka: \tema}
\author{Björn Regnell}
\institute{Datavetenskap, LTH}
\date{Lp1-2, HT 2015}
 
% Package to draw UML diagrams in file: pgf-umlcd.sty 
% https://github.com/xuyuan/pgf-umlcd
% http://people.irisa.fr/Martin.Quinson/blog/2010/1105/UML_class_diagrams_in_tikz/index.html
\usepackage[simplified]{pgf-umlcd}  % simplified removes empty parts of a class
%\usepackage{pgf-umlcd}
% Example:
%\begin{tikzpicture}
%\begin{class}[]{Robot}{1,3}
%  \attribute{name : String}
%  \operation{walk(distance: int) : void}
%\end{class}
%\begin{class}[]{TalkingRobot }{1,0}
% \inherit{Robot}
%  \operation{speak(s: String) : void}
%\end{class}
%\end{tikzpicture}

%%%%%%%%%%%%%%%%%%%%%%%%%%%%%%%%%%%%%%

\begin{document}

\frame{\titlepage}
\setnextsection{\vecka}
\section[Vecka \vecka: \tema]{\tema}
\frame{\tableofcontents}

\subsection{Att göra denna vecka}
\begin{Slide}{Att göra i Vecka \vecka: Förstå \href{https://sv.wikipedia.org/wiki/Arv_\%28programmering\%29}{arv}. Repetera inför kontrollskrivning.}
\begin{enumerate}
\item Läs följande kapitel i kursboken: ??? \\  
Begrepp: registrering, equals, arv.
\item Gör övning 7: registrering
\item Träffas i samarbetsgrupper och hjälp varandra 
\item Gör Lab 6: Implementera Turtle och ColorTurtle.
\end{enumerate}
\end{Slide}

\subsection{Grundläggande terminologi för arv}
\begin{Slide}{Vad är arv? Motivering och terminologi}\footnotesize
\begin{itemize}
\item Med hjälp av \Emph{arv} mellan klasser kan man göra så att en klass \Emph{ärver} (''får med sig'') innehåll från en \textit{annan} klass.
\item Varför vill man det? 
\begin{enumerate}\footnotesize
\item Dela upp ansvar mellan klasser och bryta ut gemensamma delar så att man slipper duplicerad kod.
\item Skapa en klassificering av objekt utifrån relationen  \Emph{X är en Y}.  \\ Exempel: En gurka är en grönsak. En cykel är ett fordon. 
\end{enumerate}
\item Nyckelordet \Key{extends} används för att ange arv i Java. \\ Exempel:   \code|class TalkingRobot extends Robot|
\item Klassen som ärver (utökar) kallas \Emph{subklass}
\item Klassen som blir utökad kallas \Emph{superklass} (även \textit{basklass})
\item Läs mer om arv \Eng{inheritance} här: \scriptsize \url{https://sv.wikipedia.org/wiki/Arv\_\%28programmering\%29} 

\end{itemize}
\end{Slide}

\end{document}