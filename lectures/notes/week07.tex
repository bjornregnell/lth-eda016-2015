\documentclass{lecturenotes}

\renewcommand{\vecka}{7}
\newcommand{\tema}{Arv}

\setbeamertemplate{footline}[frame number]
\title[Föreläsningsanteckningar EDA016, 2015]{EDA016 Programmeringsteknik för D}
\subtitle{Läsvecka \vecka: \tema}
\author{Björn Regnell}
\institute{Datavetenskap, LTH}
\date{Lp1-2, HT 2015}

%%%%%%%%%%%%%%%%%%%%%%%%%%%%%%%%%%%%%%

\begin{document}

\frame{\titlepage}
\setnextsection{\vecka}
\section[Vecka \vecka: \tema]{\tema}
\frame{\tableofcontents}

\subsection{Att göra denna vecka}
\begin{Slide}{Att göra i Vecka \vecka: Förstå arv. \\Repetera inför kontrollskrivning.}
\begin{enumerate}
\item Läs följande kapitel i kursboken:\\ 9.1, 9.3, 9.7-9.9, 11.2, 12.6 \\  
Begrepp: arv, subklass, superklass, \code{extends},  \code{abstract}, operatorn \code{instanceof}, \code{Object},  \code{super},  \code{equals},
\item Gör övning 7: registrering
\item Träffas i samarbetsgrupper och hjälp varandra 
\item Gör Lab 6: Implementera Turtle och ColorTurtle.
\item Plugga inför \Alert{obligatorisk} kontrollskrivning.
\end{enumerate}
\end{Slide}

\subsection{Grundläggande terminologi för arv}
\begin{Slide}{Vad är arv? Motivering och terminologi}\footnotesize
\begin{itemize}
\item Med hjälp av \Emph{arv} mellan klasser kan man göra så att en klass \Emph{ärver} (''får med sig'') innehållet i en \textit{annan} klass.
\item Varför vill man det? 
\begin{enumerate}\footnotesize
\item Dela upp ansvar mellan klasser och bryta ut gemensamma delar så att man slipper duplicerad kod.
\item Skapa en klassificering av objekt utifrån relationen  \Emph{X är en Y}.  \\ Exempel: En gurka är en grönsak. En cykel är ett fordon. 
\end{enumerate}
\item Nyckelordet \Key{extends} används för att ange arv i Java. \\ Exempel:   \code|class TalkingRobot extends Robot|
\item Klassen som ärver (utökar) kallas \Emph{subklass}
\item Klassen som blir utökad kallas \Emph{superklass} (även \textit{basklass})
\item Läs mer om arv \Eng{inheritance} här: \scriptsize \url{https://sv.wikipedia.org/wiki/Arv\_\%28programmering\%29} 
\end{itemize}
\end{Slide}

\begin{Slide}{Exempel: \href{https://www.youtube.com/watch?v=VXa9tXcMhXQ}{Robot}}
\Emph{Krav}: \\ Det ska finnas två sorters robotar: \code{Robot} och \code{TalkingRobot}. Båda kan arbeta, men bara den senare kan prata.
\\ \vspace{1em}
En möjlig \Emph{design} om vi \textit{inte} använder arv:

\begin{ClassSpec}{Robot}
/** Arbeta */
public void work();
\end{ClassSpec}

\begin{ClassSpec}{TalkingRobot}
/** Arbeta */
public void work();

/** Prata */
public void talk();
\end{ClassSpec}
\end{Slide}

\begin{Slide}{Utökning med copy-paste ger duplicerad kod}\footnotesize
\lstinputlisting[language=Java, basicstyle=\ttfamily\tiny\selectfont, numberstyle=, numbers=left,]{../examples/eclipse-ws/lecture-examples/src/week07/robotduplicate/RobotNoInheritance.java}
\end{Slide}

\begin{Slide}{Utökning med arv -- vi slipper duplicera kod}\footnotesize
\lstinputlisting[language=Java, basicstyle=\ttfamily\tiny\selectfont, numberstyle=, numbers=left,]{../examples/eclipse-ws/lecture-examples/src/week07/robotinherit/RobotInheritance.java}
\end{Slide}

\begin{Slide}{Illustrera arvsrelationer i diagram}
\begin{center}
\begin{tikzpicture}
\node (Robot) [umlclass, rectangle split parts = 2]  {
            \textbf{\centerline{Robot}}
            \nodepart[]{second}work(): void
        };
\node (TalkingRobot) [umlclass, rectangle split parts = 2, below = of Robot]  {
            \textbf{\centerline{TalkingRobot}}
            \nodepart[]{second}talk(): void
        };        
\draw[umlarrow] (TalkingRobot.north) -- ++(0,0.8) -| (Robot.south);        
\end{tikzpicture}
\end{center}
\vspace{2em} 
\centerline{\scriptsize\href{https://en.wikipedia.org/wiki/Class_diagram}{Class Diagram}, Unified Modelling Language (\href{https://sv.wikipedia.org/wiki/Unified\_Modeling\_Language}{UML}), Ankboken sid 147.}
\end{Slide}

\subsection{Klassificering och abstrakta klasser}
\begin{Slide}{Klassificering av robotar}
\begin{center}
\begin{tikzpicture}
\node (AbstractRobot) [umlclass, rectangle split parts = 2, xshift=-2cm]  {
            \textit{\textbf{\centerline{AbstractRobot}}}
            \nodepart[]{second}work(): void
        };
        
\node (MuteRobot) [umlclass, rectangle split parts = 2, below = of Robot, left = of TalkingRobot, yshift=0.05mm]  {
            \textbf{\centerline{MuteRobot}}
            \nodepart[]{second} \vspace{1em}
        };  
                
\node (TalkingRobot) [umlclass, rectangle split parts = 2, below = of AbstractRobot, xshift=2cm, yshift=-0.3mm]  {
            \textbf{\centerline{TalkingRobot}}
            \nodepart[]{second}talk(): void
        };        
\draw[umlarrow] (TalkingRobot.north) -- ++(0,0.5) -| (AbstractRobot.south);    
\draw[umlarrow] (MuteRobot.north) -- ++(0,0.5) -| (AbstractRobot.south);            
\end{tikzpicture}
\end{center}
\vspace{0.25em}\pause
\begin{itemize}
\item \Emph{Abstrakta klasser} används som bas för klassificering
\item Deklaration i Java: \code{abstract class ClassName}
\item Försök att instantiera abstrakta klasser ger kompileringsfel. 
\item \code{MuteRobot} är en \Emph{konkret klass} som kan instantieras och ärver \code{work()}-metoden från \code{AbstractRobot}
\end{itemize}
\end{Slide}

\begin{Slide}{Exempel: klassificeringsmodell  i Java}
\lstinputlisting[language=Java, basicstyle=\ttfamily\tiny\selectfont, numberstyle=, numbers=left,]{../examples/eclipse-ws/lecture-examples/src/week07/robotclassification/AbstractRobot.java}

\lstinputlisting[language=Java, basicstyle=\ttfamily\tiny\selectfont, numberstyle=, numbers=left,]{../examples/eclipse-ws/lecture-examples/src/week07/robotclassification/TalkingRobot.java}

\lstinputlisting[language=Java, basicstyle=\ttfamily\tiny\selectfont, numberstyle=, numbers=left,]{../examples/eclipse-ws/lecture-examples/src/week07/robotclassification/MuteRobot.java}
\end{Slide}

\begin{Slide}{Exempel: test av klassificeringsmodell}
\lstinputlisting[language=Java, basicstyle=\ttfamily\tiny\selectfont, numberstyle=, numbers=left,]{../examples/eclipse-ws/lecture-examples/src/week07/robotclassification/TestRobots.java}
\end{Slide}

\subsection{Operatorn \texttt{instanceof}}
\begin{Slide}{Operatorn \texttt{instanceof}}
\lstinputlisting[language=Java, basicstyle=\ttfamily\tiny\selectfont, numberstyle=, numbers=left,]{../examples/eclipse-ws/lecture-examples/src/week07/robotclassification/TestInstanceOf.java}
\end{Slide}

\subsection{Klassen \texttt{Object}}
\begin{Slide}{Alla objekt är implicita instanser av klassen \code{Object}}
\begin{center}
\begin{tikzpicture}
\node (Object) [umlclass, rectangle split parts = 2]  {
            \textbf{\centerline{Object}}
            \nodepart[]{second}toString(): String
        };
        
\node (AbstractRobot) [umlclass, below = of Object, rectangle split parts = 2,]  {
            \textit{\textbf{\centerline{AbstractRobot}}}
            \nodepart[]{second}work(): void
        };
        
\node (MuteRobot) [umlclass, rectangle split parts = 2, left = of TalkingRobot, below = of AbstractRobot, xshift=-2cm]  {
            \textbf{\centerline{MuteRobot}}
            \nodepart[]{second} \vspace{1em}
        };  
                
\node (TalkingRobot) [umlclass, rectangle split parts = 2, below = of AbstractRobot, xshift=2cm]  {
            \textbf{\centerline{TalkingRobot}}
            \nodepart[]{second}talk(): void
        };        
\draw[umlarrow] (TalkingRobot.north) -- ++(0,0.5) -| (AbstractRobot.south);    
\draw[umlarrow] (MuteRobot.north) -- ++(0,0.5) -| (AbstractRobot.south);         
\draw[umlarrow] (AbstractRobot.north) -- ++(0,0.5) -| (Object.south);               
\end{tikzpicture}
\end{center}
\end{Slide}

\subsection{\texttt{protected} och \texttt{super}}
\begin{Slide}{\texttt{protected} och \texttt{super}}
\lstinputlisting[language=Java, basicstyle=\ttfamily\tiny\selectfont, numberstyle=, numbers=left,]{../examples/eclipse-ws/lecture-examples/src/week07/superconstructor/Robot.java}

\lstinputlisting[language=Java, basicstyle=\ttfamily\tiny\selectfont, numberstyle=, numbers=left,]{../examples/eclipse-ws/lecture-examples/src/week07/superconstructor/TalkingRobot.java}
\end{Slide}

\begin{Slide}{Test av \texttt{protected} och \texttt{super}}
\lstinputlisting[language=Java, basicstyle=\ttfamily\tiny\selectfont, numberstyle=, numbers=left,]{../examples/eclipse-ws/lecture-examples/src/week07/superconstructor/TestSuperConstructor.java}
\begin{verbatim}
Wall-E is working.
C3PO shall not harm humans.
C3PO is working.
\end{verbatim}
\end{Slide}

\begin{Slide}{Attribut i UML-diagram}
\begin{center}
\begin{tikzpicture}
\node (Robot) [umlclass, rectangle split parts = 3]  {
            \textbf{\centerline{Robot}}
            \nodepart[]{second}name: String
            \nodepart[]{third}work(): void
       };
\node (TalkingRobot) [umlclass, rectangle split parts = 3, below = of Robot]  {
            \textbf{\centerline{TalkingRobot}}
            \nodepart[]{second} \vspace{1em}            
            \nodepart[]{third}talk(): void
        };        
\draw[umlarrow] (TalkingRobot.north) -- ++(0,0.8) -| (Robot.south);        
\end{tikzpicture}
\end{center}

\end{Slide}

\end{document}