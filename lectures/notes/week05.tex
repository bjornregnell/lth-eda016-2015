\documentclass{lecturenotes}

\renewcommand{\vecka}{5}
\newcommand{\tema}{Tecken, strängar och slumptal}

\setbeamertemplate{footline}[frame number]
\title[Föreläsningsanteckningar EDA016, 2015]{EDA016 Programmeringsteknik för D}
\subtitle{Läsvecka \vecka: \tema}
\author{Björn Regnell}
\institute{Datavetenskap, LTH}
\date{Lp1-2, HT 2015}
 
\begin{document}

\frame{\titlepage}
\setnextsection{\vecka}
\section[Vecka \vecka: \tema]{\tema}
\frame{\tableofcontents}

%%%%%%%%%%%%%%%%%%%%%%%%%%%%%%%%%%%%%%

\subsection{Att göra denna vecka}
\begin{Slide}{Att göra i Vecka \vecka: Förstå aritmetiska och logiska uttryck, använda klasser mha klass-specifikationer}
\begin{enumerate}
\item Läs följande kapitel i kursboken: 11, 7.9, 6.10, 7.7, 7.4, 7.12 \\  
Begrepp: sträng, toString, StringBuilder, slumptal, Random
\item Gör övning 5: Klasser, slumptal
\item Gör gammal kontrollskrivning \& rätta i samarbetsgrupper 
\item Gör Lab 4: implementera square
\end{enumerate}
\end{Slide}

\begin{Slide}{Hur använda föreläsningsexempel}
\begin{itemize}
\item Föreläsninsgexempel för användning i Eclipse finns här: \\ {\scriptsize
  \url{https://github.com/bjornregnell/lth-eda016-2015/tree/master/lectures/examples/eclipse-ws} }
\item Följ instruktionerna i \href{https://github.com/bjornregnell/lth-eda016-2015/blob/master/lectures/examples/eclipse-ws/README.md}{README.md} för att öppna exemplen i ett eget workspace
\item Ändra, prova varianter, gör nya testprogram, etc. Det är genom att \Alert{aktivt koda} som du lär dig!
\end{itemize}
\end{Slide}


\Subsection{Tecken}

\begin{Slide}{Specialtecken}
Some \href{https://docs.oracle.com/javase/tutorial/java/data/characters.html}{Java Escape Sequences}:
\begin{itemize}
\item Ny rad: \verb+\n+
\item Tab: \verb+\t+
\item Citationstecken: \verb+\'+
\item Tab: \verb+\"+
\item Godtyckligt unicode-tecken:  \verb+\u03BB+ 
\end{itemize}
\end{Slide}

\begin{Slide}{Tecken}
\lstinputlisting[language=Java, basicstyle=\ttfamily\tiny\selectfont, numberstyle=, numbers=left,]{../examples/eclipse-ws/lecture-examples/src/week05/ShowCharacters.java}
\end{Slide}

\begin{Slide}{Klassen Character}
Klassen \href{https://docs.oracle.com/javase/8/docs/api/java/lang/Character.html}{Character} innehåller många användbara metoder, t.ex.:
\begin{ClassSpec}{Character}
/** Determines if the specified character is a digit. */
static boolean isDigit(char ch);

/** Determines if the specified character is a letter. */
static boolean isLetter(char ch);

/** Determines if the specified character is white space. */
static boolean isWhitespace(char ch);

/** Determines if the specified character is a uppercase character. */
static boolean isUpperCase(char ch);

/** Determines if the specified character is a lowercase character. */
static boolean isLowerCase(char ch);

/** Converts the character argument to uppercase */
static char toUpperCase(char ch);

/** Converts the character argument to lowercase */
static char toLowerCase(char ch);
\end{ClassSpec}
\end{Slide}

\Subsection{Strängar}
\begin{Slide}{\code{String} och \code{StringBuilder}}
\footnotesize
Standardklasser (i paketet \code{java.lang}, behöver inte importeras):
\begin{tabular}{lp{8cm}}
\code+String+ & Beskriver en följd av tecken. Tecknen kan läsas av men inte ändras. Med operatorn \code!+! konkatenerar man (slår ihop) två strängar, eller en sträng med ett talvärde. Då bildas ett nytt strängobjekt.\\
\code+StringBuilder+ & En följd av tecken som kan läsas av \emph{och} ändras.
\end{tabular}

\begin{Code}
String s1 = "En text";
String s2 = "en text till";
String result = s1 + " och " + s2;

int x = 10;
int y = 30;
String s1 = "Summan är " + x + y;
String s2 = "Summan är " + (x + y);
\end{Code}
\end{Slide} 

\begin{Slide}{Viktiga operationer på \code{String}}

\begin{ClassSpec}{String}
String();                   // skapar en tom sträng
                            // (kan också skrivas "")
int length();               // antalet tecken 
char charAt(int pos);       // tecknet på plats pos
boolean equals(Object s);   // true om innehållet i 
                            // aktuell sträng är lika 
                            // med innehållet i s
int compareTo(String s);    // jämför aktuell sträng med s
int indexOf(char ch);       // index för den första
                            // förekomsten av ch, -1
                            // om ch inte finns
String substring(int start, // ny sträng med tecknen 
                 int end);  // med index [start, end)
\end{ClassSpec}

\begin{Code}
"Java".compareTo("Java") == 0
"java".compareTo("javac") < 0
"java".compareTo("Java") > 0
"java".compareTo("jazz") < 0
\end{Code}

\end{Slide} 

\begin{Slide}{Användning av \code{String}, 1}
\begin{Code}
public class Text {
    private String s;
    
    public Text(String s) {
        this.s = s;
    }

    /** Tar reda på antalet blanktecken i strängen */
    public int getNbrSpaces() {
        int spaces = 0;
        for (int i = 0; i < s.length(); i++) {
            if (s.charAt(i) == ' ') {
                spaces++;
            }
        }
        return spaces;
    }
}
\end{Code}
\end{Slide} 

\begin{Slide}{Användning av \code{String}, 2}
\begin{Code}
/** Tar reda på index för den första förekomsten av tecknet
    ch i texten, -1 om inget sådant tecken finns */
public int indexOf(char ch) {
    int i = 0;
    while (i < s.length() && s.charAt(i) != ch) {
        i++;
    }
    return (i < s.length()) ? i : -1;
}
\end{Code}
\end{Slide} 

\begin{Slide}{Användning av \code{String}, 3}
\begin{Code}
/** Tar reda på det första ordet i texten. Ett ord är en 
    följd av tecken som inte är blanka */
public String firstWord() {
    int start = 0;
    while (start < s.length() &&
           Character.isWhitespace(s.charAt(start))) {
        start++;
    }
    int end = start;
    while (end < s.length() && 
           ! Character.isWhitespace(s.charAt(end))) {
        end++;
    }
    return s.substring(start, end);
}
\end{Code}

\end{Slide} 

\begin{Slide}{\code{StringBuilder}}
Skapa, ta reda på längd och tecken (som \code{String}):
\begin{ClassSpec}{StringBuilder}
StringBuilder();         // tom strängbuffert
StringBuilder(String s); // kopia av s
int length();            // antalet tecken
char charAt(int pos);    // tecknet på plats pos
String toString(); // skapar ett String-objekt med
                   // samma innehåll som denna 
                   // strängbuffert
\end{ClassSpec}
\end{Slide} 

\begin{Slide}{Ändra innehållet i \code{StringBuilder}-objekt}
\begin{ClassSpec}{StringBuilder}
void setCharAt(int k, char ch); // ändrar tecknet på 
                                // plats k till ch
StringBuilder append(String s); // lägger till s 
                                // efter tecknen 
StringBuilder insert(int k, String s); // lägger in s 
                                // på plats k, flyttar
                                // tecknen efter
StringBuilder deleteCharAt(int k); // tar bort tecknet 
                                // på plats k
StringBuilder delete(int start, int end); // tar bort
                                // tecknen [start,end)
StringBuilder replace(int start, int end, String s); 
                                // ersätter tecknen
                                // [start,end) med s
\end{ClassSpec}
\end{Slide} 

\begin{Slide}{Användning av \code{StringBuilder}, 1}
\begin{Code}
public class MutableText {
    private StringBuilder sb;
    
    public MutableText(String s) {
        sb = new StringBuilder(s);
    }

    /** Ändrar alla små bokstäver a-z i texten till motsvarande stora */
    public void changeToUpperCase() {
        for (int i = 0; i < sb.length(); i++) {
            char ch = sb.charAt(i);
            if (ch >= 'a' && ch <= 'z') {
                sb.setCharAt(i, (char) (ch - 'a' + 'A'));
            }
        }
    }
\end{Code}
\end{Slide} 

\begin{Slide}{Användning av \code{StringBuilder}, 2}
\begin{Code}
/** Lägger in ett blanktecken efter varje punkt och kommatecken 
    i texten, dock ej efter textens sista tecken */
public void insertSpaces() {
    int i = 0;
    while (i < sb.length() - 1) {
        if (sb.charAt(i) == '.' || sb.charAt(i) == ',') {
            sb.insert(i + 1, ' ');
            i++;
        }
        i++;
    }
}
\end{Code}
\end{Slide} 

\begin{Slide}{Ta bort blanktecken från sträng}

\begin{Code}
public static String removeSpacesSlow(String s) { // "DÅLIGT"
    String result = "";
    for (int i = 0; i < s.length(); i++) {
        if (s.charAt(i) != ' ') {
            result += s.charAt(i);
        }
    }
    return result;
}

public static String removeSpaces(String s) { // "BRA"
    StringBuilder sb = new StringBuilder();
    for (int i = 0; i < s.length(); i++) {
        if (s.charAt(i) != ' ') {
            sb.append(s.charAt(i));
        }
    }
    return sb.toString();
}
\end{Code}
\end{Slide} 

\begin{Slide}{Exempel: Nananananana Batman!}
\url{https://www.youtube.com/watch?v=oDc-1zfffMw} \\ \vspace{1em}

Prova \href{https://github.com/bjornregnell/lth-eda016-2015/blob/master/lectures/examples/eclipse-ws/lecture-examples/src/week05/NanananananananaNanananananananaBatman.java}{detta exempel} på din dator och se hur snabbt det går för stora \code{n} med \code{String} versus \code{StringBuilder}:
\begin{Code}
*** MEASURING n == 16
  Singing Batman with String:        Timer measured: 0 ms
  Singing Batman with StringBuilder: Timer measured: 0 ms
\end{Code}
\end{Slide} 

\begin{Slide}{Hur snabb är din dator? Så här snabb är min i7-4790K}
\begin{Code}[basicstyle=\ttfamily\fontsize{6}{7}\selectfont]
*** MEASURING n == 1024
  Singing Batman with String:        Timed: less than 7 ms
  Singing Batman with StringBuilder: Timed: less than 1 ms
*** MEASURING n == 2048
  Singing Batman with String:        Timed: less than 24 ms
  Singing Batman with StringBuilder: Timed: less than 1 ms
*** MEASURING n == 4096
  Singing Batman with String:        Timed: less than 74 ms
  Singing Batman with StringBuilder: Timed: less than 1 ms
*** MEASURING n == 8192
  Singing Batman with String:        Timed: less than 241 ms
  Singing Batman with StringBuilder: Timed: less than 1 ms
*** MEASURING n == 16384
  Singing Batman with String:        Timed: less than 873 ms
  Singing Batman with StringBuilder: Timed: less than 1 ms
*** MEASURING n == 32768
  Singing Batman with String:        Timed: less than 4269 ms
  Singing Batman with StringBuilder: Timed: less than 1 ms
*** MEASURING n == 65536
  Singing Batman with String:        Timed: less than 17877 ms
  Singing Batman with StringBuilder: Timed: less than 1 ms
*** MEASURING n == 131072
  Singing Batman with String:        Timed: less than 78529 ms
  Singing Batman with StringBuilder: Timed: less than 2 ms
*** MEASURING n == 262144
  Singing Batman with String:        Timed: less than 376264 ms
  Singing Batman with StringBuilder: Timed: less than 4 ms  
\end{Code}
\end{Slide}

\begin{Slide}{\code{toString}}

\begin{Code}
public class Complex {
    private double re; // realdel
    private double im; // imaginärdel
    
    public Complex(double re, double im) {
        this.re = re;
        this.im = im;
    }
    
    public String toString() { //vad händer i exemplet om du tar bort denna?
        return "Complex(" + re + ", " + im + ")";
    }
}
\end{Code}


Exempel på explicit och \href{https://docs.oracle.com/javase/specs/jls/se8/html/jls-4.html#jls-4.3.2}{implicit} användning av \code{toString}:

\begin{Code}
Complex z = new Complex(1.5, 2.3);
System.out.println("z = " + z.toString());
System.out.println("z = " + z);
\end{Code}
\end{Slide} 

\begin{Slide}{Programexempel: Datakomprimering}
Användning av s.k. följdlängdskodning\\
\Emph{Krav}
\begin{enumerate} \footnotesize
 \item Man räknar hur många gånger som ett tecken förekommer i följd.
 \item Om antalet är större än 3 lagras först ett dollartecken, sedan antalet tecken, sedan tecknet. Dollartecknet fungerar som ett ''escapetecken'' som talar om att de följande tecknen ska tolkas på ett speciellt sätt. 
 \item Om antalet är mindre än eller lika med 3 lagras alla tecken.
 \item Exempel: \textsl{aabbbbbcdddeeeeeeffff} kodas som \textsl{aa\$5bcddd\$6e\$4f}. Siffrorna är inte tecknen \code{'5'}, \code{'6'} och \code{'4'} utan tecknen med Unicode-numren 5, 6 respektive 4.
\item Förenklingar: 
\begin{itemize} \footnotesize
\item Vi förutsätter att det inte finns några dollartecken i texten.
\item Vi förutsätter att det inte finns fler tecken i rad än att antalet ryms i en \code{char}-variabel (16 bitar).
\end{itemize}
\end{enumerate}
\end{Slide} 

\begin{Slide}{Compressor specifikation och design}
\begin{ClassSpec}{Compressor}
/** komprimerar klartextsträngen s med följdlängdskodning */
public static String compress(String s) 

/** dekomprimerar följdlängdskodade strängen s till klartext  */
public static String decompress(String s)
\end{ClassSpec}

\Emph{Design av \texttt{compress}}: \\ pseudokod:
\begin{itemize} 
\item så länge strängen inte är slut:
\begin{itemize} 
\item så länge alla tecken lika:\\ räkna antalet lika tecken i följd
\item om fler än 3 lika: \\ 
bygg på sträng med komprimerad följd \\
\textit{annars:} bygg på sträng med okomprimerad följd

\end{itemize}
\end{itemize}
\end{Slide}

\begin{Slide}{Komprimering}

\begin{Code}[basicstyle=\ttfamily\fontsize{7}{8}\selectfont]
public static String compress(String s) {
    StringBuilder sb = new StringBuilder();
    int i = 0;
    while (i < s.length()) {
        char ch = s.charAt(i);
        int nbrEqual = 1;
        i++;
        while (i < s.length() && s.charAt(i) == ch) {
            i++; 
            nbrEqual++;
        }
        if (nbrEqual > 3) {
            sb.append('$'); 
            sb.append((char) nbrEqual); 
            sb.append(ch);
        } else {
            for (int k = 0; k < nbrEqual; k++) { 
                sb.append(ch); 
            }
        }
    }
    return sb.toString();
}
\end{Code}
\end{Slide} 

\begin{Slide}{Dekomprimering}

\begin{Code}
public static String decompress(String s) {
    StringBuilder sb = new StringBuilder();
    int i = 0;
    while (i < s.length()) {
        char ch = s.charAt(i);
        if (ch != '$') {
            sb.append(ch);
        } else {
            i++;
            int nbrEqual = s.charAt(i);
            i++;
            ch = s.charAt(i);
            for (int k = 0; k < nbrEqual; k++) {
                sb.append(ch);
            }
        }
        i++;
    }
    return sb.toString();
}
\end{Code}
\end{Slide} 




%%%%%%%%%%%%%%%%%%%%%

\Subsection{Slumptal}
\begin{Slide}{Slumptal} \footnotesize
\begin{itemize}
\item \href{https://sv.wikipedia.org/wiki/Slumptalsgenerator}{Slumptalsgenerering} är ett viktigt område inom mjukvaruutveckling, speciellt inom kryptering, intrångsskydd, simulering och spelutveckling.
\item Slumptal får man i Java med hjälp av standardklassen \href{http://docs.oracle.com/javase/8/docs/api/java/util/Random.html}{\code{java.util.Random}}
\end{itemize}

\begin{ClassSpec}{Random}
/** En slumptalsgenerator med slumptalsfröet seed */
Random(long seed);

/** En slumptalsgenerator med ett slumpmässigt 
    slumptalsfrö */
Random();

/** Slumpmässigt heltal i intervallet [0,n) */
int nextInt(int n);

/** Slumpmässigt reellt tal i intervallet [0,1.0) */
double nextDouble();
\end{ClassSpec}
\end{Slide} 

\begin{Slide}
{Användning av \code{Random}}
10 slumpmässiga heltal i intervallet \code{[1, 6]}, 10 reella tal i intervallet \code{[5.0,~15.0)}:

\begin{Code}
package week05;
import java.util.Random;

public class RandomExample {
    public static void main(String[] args) {
        Random rand = new Random();
        for (int i = 0; i < 10; i++) {
            int iRand = 1 + rand.nextInt(6);
            System.out.print(iRand + " ");
        }
        System.out.print("\n\n");
        for (int i = 0; i < 10; i++) {
            double dRand = 5 + 10 * rand.nextDouble();
            System.out.println(dRand);
        }
    }
}
\end{Code}
\end{Slide} 

\begin{Slide}{Programexempel: Tärningsspel} \footnotesize
\Emph{Krav} 
\begin{enumerate} 
\item Ett tärningsspel med två spelare ska simuleras i ett program. 
\item Den förste spelaren kastar tärningen och räknar antalet kast tills det i två kast i följd blir samma antal prickar på tärningen. 
\item Därefter kastar den andre spelaren tärningen på samma sätt. 
\item Den av spelarna som gjort minst antal kast vinner spelet. 
\item Om båda spelarna gjort samma antal kast kastar de båda igen tills någon av dem vunnit. 
\item När spelet är klart ska namnet på vinnaren skrivas ut.

\end{enumerate}

\Emph{Design}:  Dela upp koden i dessa klasser: 
\begin{itemize}
\item \code{Die} har hand om data och operationer för en \Emph{tärning}
\item \code{Player} har hand om data och operationer för en \Emph{spelare}
\item \code{DiceGame} \Emph{genomför ett spel}
\end{itemize}
\end{Slide} 

\begin{Slide}
{\code{main}-metod som genomför spelet}
\Emph{Test}: Ett huvudprogram som genomför spelet:
\begin{Code}
public class PlayGame {
    public static void main(String[] args) {
        Player p1 = new Player("Kim");
        Player p2 = new Player("Robin");
        DiceGame game = new DiceGame(p1, p2);
        Player winner = game.play();
        System.out.println(winner.getName() +  " vann");
    }
}
\end{Code}
Testresultat: Utskrift av vinnarens namn.
\end{Slide} 

\begin{Slide}
{Klassen \code{Die}, en tärning}
\begin{Code}
import java.util.Random;
public class Die {
    private static Random rand = new Random();
    private int pips;

    /** Skapar en tärning */
    public Die() {
        roll(); // så att pips får ett initialt värde 1..6
    }
    
    /** Kastar tärningen */
    public void roll() {
        pips = 1 + rand.nextInt(6);
    }
    
    /** Tar reda på resultatet av det senaste kastet */
    public int getResult() {
        return pips;
    }
}
\end{Code}
\end{Slide} 

\begin{Slide}
{Specifikation av \code{Player}, en spelare}
Båda spelarna ska spela med samma tärning, så de måste få reda på ''utifrån'' vilken tärning de ska använda:

\begin{ClassSpec}{Player}
/** Skapar en spelare med namnet name */
Player(String name);

/** Spelaren kastar tärningen die tills det blir 
    två lika i följd, returnerar antalet kast */
int play(Die die);

/** Tar reda på spelarens namn */
String getName();
\end{ClassSpec}
\end{Slide} 

\begin{Slide}
{Klassen \code{DiceGame}, en spelomgång, 1}
\begin{Code}
public class DiceGame {
    private Player player1;
    private Player player2;
    private Die die;
    
    /** Skapar ett spel som spelas mellan spelarna
        player1 och player2 */
    public DiceGame(Player player1, Player player2) {
        this.player1 = player1;
        this.player2 = player2;
        die = new Die();
    }
    
    /** Genomför en spelomgång, returnerar vinnaren */
    public Player play() {
        // ... nästa bild
    }
}
\end{Code}
\end{Slide} 

\begin{Slide}
{Klassen \code{DiceGame}, en spelomgång, 2}
\begin{Code}
public class DiceGame {
    ...
    
    /** Genomför en spelomgång, returnerar vinnaren */
    public Player play() {
        int p1Rolls = player1.play(die);
        int p2Rolls = player2.play(die);
        while (p1Rolls == p2Rolls) {
            p1Rolls = player1.play(die);
            p2Rolls = player2.play(die);
        }
        return (p1Rolls < p2Rolls) ? player1 : player2;  //villkors-uttryck
    }
}
\end{Code}
\end{Slide} 

\begin{Slide}{Villkorsuttryck, \code{? :}}
\begin{Code}
logiskt uttryck ? uttryck1 : uttryck2
\end{Code}

Exempel:
\begin{Code}
int i = 1;
int j = 2;
int result = (i == 1) ? i + 5 : j + 5; // result = 6
return (i > j) ? i : j;                // 2 returneras
\end{Code}

Man kan klara sig utan villkorsuttryck:
\begin{Code}
int result;
if (i == 1) {
    result = i + 5;
} else {
    result = j + 5;
}
\end{Code}
\end{Slide} 

\begin{Slide}{Implementation av \code{Player.play}}
\begin{Code}
public class Player {
    ...
    
    /** Spelaren kastar tärningen die tills det blir 
        två lika i följd, returnerar antalet kast */
    public int play(Die die) {
        die.roll();
        int prevResult = die.getResult();
        die.roll();
        int result = die.getResult();
        int nbrRolls = 2;
        while (result != prevResult) {
            die.roll();
            nbrRolls = nbrRolls + 1;
            prevResult = result;
            result = die.getResult();
        }
        return nbrRolls;
    }
}
\end{Code}
\end{Slide} 


%%%%%%%%%%%%%%%%%%%%%%%%%%%%%%%%%%%%%%
\subsection{do-while}
\begin{Slide}
{Klassen \code{DiceGame} med do-while}
\begin{Code}
public class DiceGame {
    ...
    
    /** Genomför en spelomgång, returnerar vinnaren */
    public Player play() {
        int p1Rolls;
        int p2Rolls;
        do  {
            p1Rolls = player1.play(die);
            p2Rolls = player2.play(die);
        } while (p1Rolls == p2Rolls)
        return (p1Rolls < p2Rolls) ?  player1 : player2;
    }
}
\end{Code}
\end{Slide} 

%%%%%%%%%%%%%%%%%%%%%%%%%%%%%%%%%%%%%%
\subsection{Formatering och utskrift}
\begin{Slide}{Formatering}
Automatisk formatering vid utskrift:
\begin{Code}
int sum = 209;
System.out.println("Summan är " + sum);
\end{Code}
Formatering utan utskrift:
\begin{Code}
int sum = 209;
String s1 = "" + sum;            // "209"
String s2 = String.valueOf(sum); // "209"
\end{Code}
Se även \href{http://stackoverflow.com/questions/18730978/how-do-i-use-system-out-printf}{System.out.printf}
\end{Slide} 

\begin{Slide}{Utskrift}
Metoder i \href{http://docs.oracle.com/javase/8/docs/api/java/io/PrintWriter.html}{\code{PrintWriter}} som kan användas på \code{System.out}:
\begin{ClassSpec}{PrintWriter}
void print(String s);   // skriv strängen s
void println(String s); // "print line", som print men avsluta med övergång till ny rad
void println();         // enbart ny rad
void flush();           // skicka upplagrade utskrifter till skärmen
\end{ClassSpec}

Formatering av utskrift med \code{printf}, se ankboken 7.9, s 110:
\begin{Code}
for (int i = 2; i <= 5; i++) {
    double r = Math.sqrt(i);
    System.out.printf("%5d...%6.3f%n", i, r);
}
\end{Code}
\end{Slide} 

\begin{Slide}{Utskrift på fil}

\begin{Code}
import java.util.Random;
import java.io.PrintWriter;
import java.io.File;
import java.io.FileNotFoundException;

public class TryCatchExample {
    public static void main(String[] args) {
        PrintWriter out = null;
        try {
            out = new PrintWriter(new File("random.txt"));
        } catch (FileNotFoundException e) {
            System.err.println("Filen random.txt kunde inte öppnas");
            System.exit(1);
        }
        //... utskrifter här med System.out.print hamnar på filen
    }
}
\end{Code}
\end{Slide} 

%%%%%%%%%%%%%%%%%%%%%%%%%%%%%%%%%%%%%%%%

\subsection{Switch}
\begin{Slide}{Switch-sats}
I stället för en sekvens av if ... else if ... else if ... kan man använda en \code{switch}-sats. \Alert{Glöm inte} \code{break}!
\begin{Code}[basicstyle=\ttfamily\fontsize{7}{8}\selectfont]
    switch (w.getKey()) {
    case 'a':
        turtle.rotate(5);
        break;
    case 's':
        turtle.rotate(-5);
        break;
    case ' ':
        turtle.forward(10);
        break;
    case 'r':
        int someRandomX = (int) (Math.random() * 100 + 1);
        int someRandomY = (int) (Math.random() * 100 + 1);
        turtle.moveTo(someRandomX, someRandomY);
        break;
    default:
        break;
    }
\end{Code}
Se hela koden här i \href{https://github.com/bjornregnell/lth-eda016-2015/blob/master/lectures/examples/eclipse-ws/cs_pt/src/se/lth/cs/pt/window/SpriteTest.java}{SpriteTest.java}
\end{Slide} 
            
\begin{Slide}{Broken Switch}
Det blir \href{https://github.com/bjornregnell/lth-eda016-2015/blob/master/lectures/examples/eclipse-ws/lecture-examples/src/week05/BrokenSwitch.java}{tokigt} om man \href{http://stackoverflow.com/questions/2710300/why-do-we-need-break-after-case-statements}{glömmer} \Key{break}:
\lstinputlisting[language=Java, basicstyle=\ttfamily\tiny\selectfont, numberstyle=, numbers=left,]{../examples/eclipse-ws/lecture-examples/src/week05/BrokenSwitch.java}
\end{Slide}            

\end{document}