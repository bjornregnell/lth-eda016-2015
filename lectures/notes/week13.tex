\documentclass{lecturenotes}

\renewcommand{\vecka}{13}
\newcommand{\tema}{Designexempel}

\setbeamertemplate{footline}[frame number]
\title[Föreläsningsanteckningar EDA016, 2015]{EDA016 Programmeringsteknik för D}
\subtitle{Läsvecka \vecka: \tema}
\author{Björn Regnell}
\institute{Datavetenskap, LTH}
\date{Lp1-2, HT 2015}

%%%%%%%%%%%%%%%%%%%%%%%%%%%%%%%%%%%%%%

\begin{document}

\frame{\titlepage}
\setnextsection{\vecka}
\section[Vecka \vecka: \tema]{\tema}
\frame{\tableofcontents}

\subsection{Att göra denna vecka}
\begin{Slide}{Att göra i Vecka \vecka: Studera designexempel.}

\begin{enumerate}
\item Läs följande kapitel i kursboken:  \\  
  
\item Gör extraövningar (inkl. kolla på lösningsförslag) \\ {\scriptsize \url{http://fileadmin.cs.lth.se/cs/Education/EDA016/exercises/extraexercises.pdf}}
\item Träffas i samarbetsgrupper och hjälp varandra 
\item Diskutera inlämningsuppgiftsval med handledare 
\item Gör Grupplabb 11: Image Filters
\end{enumerate}
\end{Slide}

\subsection{Repetition: Vad är en algoritm?}
\begin{Slide}{Repetition: Vad är en algoritm? }\footnotesize
En \href{https://sv.wikipedia.org/wiki/Algoritm}{algoritm} är en stegvis beskrivning av hur man löser ett problem. \\ 
\vspace{1em}
Problemlösningsprocessens olika steg (inte nödvändigtvis i denna ordning): 
\begin{enumerate}
\item identifiera (del)\Emph{problemet}
\item Kom på en \Emph{lösningsidé}
\item Formulera en \Emph{stegvis beskrivning} som löser problemet
\item Implementera en \Emph{körbar lösning} i ''riktig'' kod
\end{enumerate}
Det krävs ofta \Emph{kreativitiet} i alla steg ovan  -- även i att \Emph{känna igen} problemet.
\end{Slide}

\subsection{Design av mjukvara}
\begin{Slide}{Delar i designprocessen för utveckling av mjukvara}
\begin{itemize}
\item Krav: Varför? Vad? \\ Intressenter, önskelmål, produktstartegier, beslut
\item Arkitektur: struktur och principiell design
\item Design: Hur? \\ Uppdelning i delproblem, vilka klasser? vilka API?
\item Implementation: Hur? \\ 
Algoritmer, kod, implementera API
\item Testning: Är det rätt kvalitet? \\ Enhetstest, Modultest, Systemtest, Acceptanstest
\item Hantera byggprocessen och olika versioner
\item Driftsättning \Eng{Deployment} 
\item Drift \Eng{Operation}
\item Support och återkoppling
\end{itemize}
\end{Slide}

\Subsection{Inbjuden gäst: Patrik Persson lajvkodar androidapp}
\begin{Slide}{Designexempel: Skriv en app för Andorid}
\begin{itemize}
\item Med de kunskaper ni tillgodogör er i denna kurs är det hyffsat lätt att komma i gång med utveckling av mobilappar i den integrerade utvecklingsmiljön \href{https://en.wikipedia.org/wiki/Android_Studio}{Android Studio}.
\item Läs mer \href{http://techworld.idg.se/2.2524/1.602344/premiar-for-android-studio}{på techworld} och \href{http://developer.android.com/develop/index.html}{på officiella hemsidan}.
\item Inbjuden gästföreläsare Patrik Persson lajvkodar androidapp i Android Studio...
\end{itemize}
\begin{center}
\includegraphics[width=0.6\textwidth]{img/android-studio}
\end{center}
\end{Slide}


\begin{Slide}{Grumligtlådan}
\begin{tabular}{r|l}
\#Lappar  & Ämne                         \\ \hline
6  & \Emph{StringBuilder}\\
3  & \Emph{Vektorer, ArrayList}\\
2  & \Emph{Implementering och användning av klasser}\\
2  & \Emph{Sorteringsalgoritmer}\\
2  & Static\\
1 & Arv\\
1  & Generics\\
1  & for-each-sats\\
1  & Flera metoder med samma namn\\
1  & Matris\\
1  & När du säger "Java" exakt vad menar du då?\\
1  & Iterator\\
1 & Volatile Image\\
\end{tabular}
\end{Slide}

\begin{Slide}{Övning: Dictionary}\footnotesize
Implementera denna klass som har hand om en ordlista. \\Använd en vektor \code{String[] words} för att spara orden.
\begin{ClassSpec}{Dictionary}
/** Skapar en ny ordlista */
Dictionary();

/** Sätt in ett nytt ord på rätt plats i listan */
void insertWord(String w);

/** Returnerar listans ord som, skilda med mellanslag */
String toString();

/** Returnerar true om ordet finns i listan, annars false */
boolean contains(String word);
\end{ClassSpec}
\vspace{1em}
Extraövning: Byt attributrepresentationen till \code{ArrayList<String>}
\end{Slide}

\end{document}