\documentclass{lecturenotes}

\renewcommand{\vecka}{2}

\title[Föreläsningsanteckningar EDA016, 2015]{EDA016 Programmeringsteknik för D}
\subtitle{Läsvecka \vecka: Kodstruktur}
\author{Björn Regnell}
\institute{Datavetenskap, LTH}
\date{Lp1-2, HT 2015}
 
\begin{document}

\frame{\titlepage}
\setnextsection{2}
\section[Vecka \vecka: Kodstruktur]{Kodstruktur}
\frame{\tableofcontents}

%%%%%%%%%%%%%%%%%%%%%%%%%%%%%%%%%%%%%%
\subsection{Algoritmer och loopar}
\subsubsection{Algoritmer, SWAP, repetition (loopar), MIN/MAX}

\begin{Slide}{Vad är en algoritm?}
En \href{https://sv.wikipedia.org/wiki/Algoritm}{algoritm} är en sekvens av instruktioner\\ som beskriver hur man löser ett problem \\
\vspace{2em}
Exempel: \\ ~matrecept \\ \pause ~uppdatera highscore i ett spel \\ ~...
\begin{tikzpicture}[overlay]
\node[xshift=0.8\textwidth, scale=1.6] at (0,0) {\includegraphics[width=0.25\textwidth]{img/highscore}};
\end{tikzpicture}
\end{Slide}

\begin{Slide}{Algoritm-exempel: Highscore}
\Emph{Problem}: Uppdatera high-score i ett spel \\ \vspace{1em}

\Emph{Varför?} \pause Så att de som spelar uppmuntras att spela mer :) \\ \vspace{1em}

\Emph{Algoritm:}\pause
\begin{enumerate}
\item $points$ $\leftarrow$ poängen efter senaste spelet
\item $highscore$ $\leftarrow$ bästa resultatet innan senaste spelet
\item \Key{om} $points$ är större än $highscore$ 
\begin{enumerate}[ ~~]
\item  Skriv ''Försök igen!''
\end{enumerate}
\Key{annars}
\begin{enumerate}[ ~~]
\item  Skriv ''Grattis!''
\end{enumerate}
\end{enumerate}
\pause
\scriptsize \Alert{Hittar du buggen?}
\end{Slide}

\begin{Slide}{Algoritm-exempel: Highscore}
\lstinputlisting[language=Java, numbers=left]{../examples/terminal/highscore/HighScore.java}
Det finns en bugg i denna implementation. Vilken? \\ Fanns buggen redan i algoritmdesignen?
% Buggen är att man inte får GRATTIS om poäng == highscore vilket är tråkigt :)
\end{Slide}

\begin{Slide}{Abstraktion -- varför?}
\begin{enumerate}
\item Dela upp problem i delproblem
\item Skapa ''byggblock'' av kod som kan återanvändas
\item Dölja komplexiteten i lösningar
\item Abstraktion är själva \Emph{essensen i all programmering}
\end{enumerate}
\begin{lstlisting}
    public static void main(String[] args){
    	askUser();
    	updateHighscore();
    }
\end{lstlisting}
Kolla hela programmet här:\\ \href{https://github.com/bjornregnell/lth-eda016-2015/blob/master/lectures/examples/terminal/highscore/HighScoreAbstraction.java}{https://github.com/bjornregnell/lth-eda016-2015} \\
i filen: \scriptsize\texttt{lectures/examples/terminal/highscore/HighScoreAbstraction.java}
\end{Slide}

\begin{Slide}{Vår första algoritmkluring: SWAP}
\Emph{Problem}: läs in och byt plats på två tal i minnet \\ \vspace{1em}
\pause
\Emph{Algoritm:}
\begin{enumerate}
\item skapa en Scanner
\item  läs in x
\item  läs in y
\item  Skriv ut x och y
\item  byt plats på värdena mellan x och y
\item  Skriv ut x och y
\end{enumerate}
\vspace{2em}
\footnotesize Varför kan det vara bra att kunna byta plats på olika värden? \\ \vspace{1em}\scriptsize
Steg 5 är egentligen en \Emph{abstraktion} av själva problemet SWAP, som inte är så lätt som det verkar och behöver delas upp i flera steg för att det ska vara rakt fram att översätta till exekverbar kod i t.ex. Java.
\end{Slide}


\begin{Slide}{Vår första algoritmkluring: SWAP}
\lstinputlisting[language=Java, numbers=left]{../examples/terminal/swap/SwapQuest.java}
\end{Slide}

\begin{Slide}{Vår första algoritmkluring: SWAP}
\lstinputlisting[language=Java, numbers=left]{../examples/terminal/swap/SwapSolution.java}
\footnotesize Övning: Rita hur minnet ser ut efter respektive raderna 7, 8, 12, 13, 14
\end{Slide}

\begin{Slide}{Mitt första program: en oändlig loop}
\begin{columns}
\begin{column}{0.8\textwidth}
\begin{verbatim}
10 print "hej"
20 goto 10
\end{verbatim}
\includegraphics[width=0.8\textwidth]{img/abc80.jpg}
\end{column}
\begin{column}{0.3\textwidth}
\pause
\begin{verbatim}
hej
hej
hej
hej
hej
hej
hej
hej
hej
hej
hej
hej
<Ctrl+C>
\end{verbatim}

\end{column}
\end{columns}
\end{Slide}

\begin{Slide}{Repetition med \Key{while}-sats}
\lstinputlisting[language=Java, numbers=left]{../examples/terminal/loops/InfiniteLoop.java}
\pause
\begin{itemize}
\item En av de saker en dator är \textit{extra} bra på är att göra samma sak om och om igen utan att tröttna! \\
Och det är ju människor \textit{extra} dåliga på :)
\item Med klockfrekvens i storleksordningen $10^9$ Hz är det ganska många instruktioner som kan göras per sekund...
\end{itemize}
\end{Slide}

\begin{Slide}{Oändlig \Key{while}-loop med räknare}
\lstinputlisting[language=Java, numbers=left]{../examples/terminal/loops/InfiniteLoopWithCounter.java}
\end{Slide}

\begin{Slide}{Ändlig \Key{while}-loop med räknare}
\lstinputlisting[language=Java, numbers=left]{../examples/terminal/loops/FiniteWhileLoopWithCounter.java}
\end{Slide}


\begin{Slide}{\Key{for}-loop med räknare}
\lstinputlisting[language=Java, numbers=left]{../examples/terminal/loops/ForLoopWithCounter.java}
Denna sats är ekvivalent med \Key{while}-satsen på föregående bild.\footnote{\scriptsize Förutom att variabeln \texttt{i} finns efter \Key{while}-satsen men \textit{inte} efter \Key{for}-satsen }
\end{Slide}

\begin{Slide}{Ändlig \Key{while}-loop med timer}
\lstinputlisting[language=Java, numbers=left]{../examples/terminal/loops/LoopWithTimer.java}
\scriptsize 
Övning: Skriv om till \Key{for}-loop och kolla om den är lika snabb som \Key{while}
\end{Slide}

\begin{Slide}{Algoritm: MIN/MAX}
\Emph{Problem}: hitta största talet \\ \vspace{1em}
\pause
\Emph{Algoritm:} 
\begin{enumerate}
\item $scan$ $\leftarrow$ en Scanner som läser det användaren skriver
\item $maxSoFar$ $\leftarrow$ ett heltal som är \textit{mindre} än alla andra heltal
\item  \Key{sålänge} det finns fler heltal att läsa: \\
~~ $x$ $\leftarrow$ läs in ett heltal med hjälp av $scan$ \\
~~ \Key{om} $x$ är större än $maxSoFar$ \\
~~~~~~ $maxSoFar$ $\leftarrow$ $x$
\item skriv ut $maxSoFar$ 
\end{enumerate}
\vspace{1em} \scriptsize 
Övning 1: Kör algoritmen med papper och penna med indata: \texttt{0~~41~~1~~45~~2~~3~~4}\\ 
Övning 2: skriv om så att algoritmen istället hittar \textit{minsta} talet.
\end{Slide}

\begin{Slide}{Övning: Implementera algoritmen MIN/MAX i Java}
\footnotesize
Några ledtrådar:
\begin{enumerate}
\item Man kan få det minsta heltalet med \href{https://docs.oracle.com/javase/8/docs/api/java/lang/Integer.html}{\lstinline{Integer.MIN_VALUE}} (negativt värde)
\item Man kan få det största heltalet med \href{https://docs.oracle.com/javase/8/docs/api/java/lang/Integer.html}{\lstinline{Integer.MAX_VALUE}}
\item Dokumentation av klassen \href{https://docs.oracle.com/javase/8/docs/api/java/util/Scanner.html}{Scanner} finns här: \url{https://docs.oracle.com/javase/8/docs/api/}
\item Man kan kolla om det finns mer att läsa med \href{https://docs.oracle.com/javase/8/docs/api/java/util/Scanner.html#hasNextInt--}{\lstinline{scan.hasNextInt()}}
\item Man läser nästa heltal med  \href{http://docs.oracle.com/javase/8/docs/api/java/util/Scanner.html#nextInt%28%29}{\lstinline{scan.nextInt()}}
\end{enumerate}
\vspace{2em}
\scriptsize Googlingstävling 1: Vem hittar först \underline{största} Double-värdet i Java? \\ Googlingstävling 2: Vem hittar först \underline{minsta} Double-värdet i Java?
% http://www.avajava.com/tutorials/lessons/how-do-i-find-the-max-and-min-values-of-primitive-types.html
\end{Slide}

%%%%%%%%%%%%%%%%%%%%%%%%%%%%%%%%%%%%%%
\subsection{Varför behövs kodstruktur?}
\begin{Slide}{Varför kodstruktur?}
\begin{itemize}
\item Nu blev denna programdel för stor och behöver delas upp...
\end{itemize}
\end{Slide}

\Subsection{Objekt}

\begin{Slide}{Objekt och referensvariabler}
% http://tex.stackexchange.com/questions/45404/asymmetric-cloud-shape-in-tikz
% http://tex.stackexchange.com/questions/44940/cloud-with-lines-as-filling-in-tikz
\begin{tikzpicture}[font=\large\sffamily]
\node[cloud, cloud puffs=15.7, cloud ignores aspect, %minimum width=5cm, minimum height=2cm,
 align=center, draw] (cloud) at (0cm, 0cm) {objekt};
\end{tikzpicture}
\end{Slide}

\end{document}