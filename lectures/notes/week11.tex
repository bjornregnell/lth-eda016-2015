\documentclass{lecturenotes}

\renewcommand{\vecka}{11}
\newcommand{\tema}{Polymorfism}

\setbeamertemplate{footline}[frame number]
\title[Föreläsningsanteckningar EDA016, 2015]{EDA016 Programmeringsteknik för D}
\subtitle{Läsvecka \vecka: \tema}
\author{Björn Regnell}
\institute{Datavetenskap, LTH}
\date{Lp1-2, HT 2015}

%%%%%%%%%%%%%%%%%%%%%%%%%%%%%%%%%%%%%%

\begin{document}

\frame{\titlepage}
\setnextsection{\vecka}
\section[Vecka \vecka: \tema]{\tema}
\frame{\tableofcontents}

\subsection{Att göra denna vecka}
\begin{Slide}{Att göra i Vecka \vecka: Förstå arv och polymorfism.}
\begin{enumerate}
\item Läs följande kapitel i kursboken:  9 \\  
Begrepp: polymorfism, klassificering, polymorfa variabler, dynamisk bindning, överskugga (override), virtuell metod, definitiva metoder och klasser
\item Gör övning 10: arv
\item Träffas i samarbetsgrupper och hjälp varandra 
\item Gör Lab 9: grupplabb TurtleRace
\end{enumerate}
\end{Slide}


\subsection{Repetition: arv}
\begin{Slide}{Repetition: Vad är arv? Motivering och terminologi}\footnotesize
\begin{itemize}
\item Med hjälp av \Emph{arv} mellan klasser kan man göra så att en klass \Emph{ärver} (''får med sig'') innehållet i en \textit{annan} klass.
\item Varför vill man det? 
\begin{enumerate}\footnotesize
\item Dela upp ansvar mellan klasser och bryta ut gemensamma delar så att man slipper duplicerad kod.
\item Skapa en klassificering av objekt utifrån relationen  \Emph{X är en Y}.  \\ Exempel 1: En gurka är en grönsak. En tomat är en grönsak. \\ Exempel 2: En cykel är ett fordon. En bil är ett fordon. 
\end{enumerate}
\item Nyckelordet \Key{extends} används för att ange arv i Java. \\ Exempel:   \code|class TalkingRobot extends Robot|
\item Klassen som ärver (utökar) kallas \Emph{subklass}
\item Klassen som blir utökad kallas \Emph{superklass} (även \textit{basklass})
\item Läs mer om arv \Eng{inheritance} här: \scriptsize \url{https://sv.wikipedia.org/wiki/Arv\_\%28programmering\%29} 
\end{itemize}
\end{Slide}

\begin{Slide}{Skydd i samband med arv}
Använd \code{protected} för synlighet bara i subklasser:
\begin{Code}
public class A {
    private int x;
    protected int y;
    public int z;
}
\end{Code}
\begin{Code}
public class B extends A {
    // här är de ärvda attributen y och z tillgängliga,
    // x är inte tillgängligt
}
\end{Code}
\begin{itemize}\footnotesize
\item Läs om skyddsregler i ankboken 9.2 och \href{https://docs.oracle.com/javase/tutorial/java/javaOO/accesscontrol.html}{officiella java tutorial}.
\end{itemize}
\end{Slide}

\begin{Slide}{Konstruktorer och arv}
Konstruktorn i subklassen måste \Alert{först} anropa superklassens konstruktor med \code{super}:
\begin{Code}[numberstyle=]
public class A {
    private int a;
    
    public A(int a){
        this.a = a;
    }
}
\end{Code}
\begin{Code}
public class B extends A {
    private int b;
    
    public B(int a, int b){
        super(a);
        this.b = b;
    }
}
\end{Code}
\end{Slide}

\begin{Slide}{Abstrakt klass}
En abstrakt klass får \Alert{inte} instansieras. \\ Vid försök blir det \Alert{kompileringsfel}:
\begin{Code}[numberstyle=]
public abstract class A {
    private int a;
    
    public A(int a){
        this.a = a;
    }
}
\end{Code}
\begin{Code}
    A a = new A(42);  // compile error: Cannot instantiate type A
\end{Code}
\begin{itemize}\footnotesize
\item Ska konstruktorer i abstrakta klasser vara \code{public} eller \code{protected}? Läs mer \href{http://stackoverflow.com/questions/260744/abstract-class-constructor-access-modifier}{på SO: abstract-class-constructor-access-modifier}
\end{itemize}
\end{Slide}


\subsection{Polymorfism}
\begin{Slide}{Polymorfism}
Polymorfism betyder \Emph{många olika skepnader}.
\begin{itemize}\footnotesize
\item Det finns flera olika slags polymorfism, bland andra:
\begin{itemize}\footnotesize
\item \Emph{Subtypning}: Variabler av en supertyp kan innehålla värden av olika subtyp. I Java används arv för att åstadkomma detta, t.ex. genom att en referensvariabel av typen \code{Shape} kan referera till \emph{olika} slags grafiska objekt, så som \code{Polygon} och \code{Circle}
\item \Emph{Parametrisk polymorfism}: En metod eller klass kan göra generisk och implementeras oberoende av vilken typ som hanteras. I java, t.ex.: \code{ArrayList<E>}
\end{itemize}
\item Läs mer här
\begin{itemize}\footnotesize
\item \href{https://sv.wikipedia.org/wiki/Polymorfism_\%28objektorienterad_programmering\%29}{svenska wikipedia}
\item \href{https://en.wikipedia.org/wiki/Polymorphism_\%28computer_science\%29}{engelska wikipedia}
\item \href{http://docs.oracle.com/javase/tutorial/java/IandI/polymorphism.html}{java tutorial}
\end{itemize}
\end{itemize}
\end{Slide}

\begin{Slide}{Exempel på polymorfism: Klassificering av mat} \footnotesize
\begin{center}

\tikzstyle{umlclass}=[rectangle, draw=black,  thick, anchor=north, text width=2cm, rectangle split, rectangle split parts = 3]
\begin{tikzpicture}[thick,scale=0.8, every node/.style={scale=0.8}]
\node (Food) [umlclass, rectangle split parts = 2, text width=3cm]  {
            \textit{\textbf{\centerline{Food}}}
            \nodepart[]{second}getWeight(): Int \\ show(): void
        };
        
\node (Vegetarian) [umlclass, rectangle split parts = 2, below = of Food,  xshift=-3cm, text width=3cm]  {
            \textit{\textbf{\centerline{Vegetarian}}}
            \nodepart[]{second} \vspace{1em}
        };  
        
\node (Cucumber) [umlclass, rectangle split parts = 2, below = of Vegetarian, xshift=-2cm]  {
            \textbf{\centerline{Cucumber}}
            \nodepart[]{second}\vspace{1em}
        }; 
        
\node (Tomato) [umlclass, rectangle split parts = 2, below = of Vegetarian, right = of Cucumber, xshift=-0.5cm]  {
            \textbf{\centerline{Tomato}}
            \nodepart[]{second}\vspace{1em}
        };           
                
\node (Animal) [umlclass, rectangle split parts = 2, below = of Food, right = of Vegetarian, xshift=1.5cm, text width=2.5cm,  text width=3cm]  {
            \textit{\textbf{\centerline{Animal}}}
            \nodepart[]{second}\small getSound(): String
        }; 
        
\node (Cow) [umlclass, rectangle split parts = 2, below = of Animal, xshift=-1.5cm]  {
            \textbf{\centerline{Cow}}
            \nodepart[]{second}\vspace{1em}
        }; 
        
\node (Pig) [umlclass, rectangle split parts = 2, below = of Animal, right = of Cow, xshift=-0.5cm]  {
            \textbf{\centerline{Pig}}
            \nodepart[]{second}\vspace{1em}
        };         
                       
\draw[umlarrow] (Animal.north) -- ++(0,0.5) -| (Food.south);    
\draw[umlarrow] (Vegetarian.north) -- ++(0,0.5) -| (Food.south);       
\draw[umlarrow] (Cow.north) -- ++(0,0.5) -| (Animal.south);            
\draw[umlarrow] (Pig.north) -- ++(0,0.5) -| (Animal.south);        
\draw[umlarrow] (Cucumber.north) -- ++(0,0.5) -| (Vegetarian.south);            
\draw[umlarrow] (Tomato.north) -- ++(0,0.5) -| (Vegetarian.south);                    
\end{tikzpicture}
\end{center}
Metoden \code{show()} förekommer \Emph{i många skepnader}, beroende på vilken konkret subklass som instansieras. Vid \emph{körtid} avgörs vilken som anropas. Detta kallas \Emph{dynamisk bindning} och metoden \code{show()} kallas \Emph{virtuell}.
\end{Slide}

\begin{Slide}{Den abstrakta klassen \code{Food}}
\lstinputlisting[language=Java, basicstyle=\ttfamily\tiny\selectfont, numberstyle=, numbers=left,]{../examples/eclipse-ws/lecture-examples/src/week11/polymorphism/Food.java}
\href{https://github.com/bjornregnell/lth-eda016-2015/blob/master/lectures/examples/eclipse-ws/lecture-examples/src/week11/polymorphism/Food.java}{lecture-examples/src/week11/polymorphism/Food.java}
\end{Slide}

\begin{Slide}{Den abstrakta klassen \code{Animal}}
\lstinputlisting[language=Java, basicstyle=\ttfamily\tiny\selectfont, numberstyle=, numbers=left,]{../examples/eclipse-ws/lecture-examples/src/week11/polymorphism/Animal.java}
\href{https://github.com/bjornregnell/lth-eda016-2015/blob/master/lectures/examples/eclipse-ws/lecture-examples/src/week11/polymorphism/Food.java}{lecture-examples/src/week11/polymorphism/Animal.java}
\end{Slide}

\begin{Slide}{Den konkreta klassen \code{Cow}}
\lstinputlisting[language=Java, basicstyle=\ttfamily\tiny\selectfont, numberstyle=, numbers=left,]{../examples/eclipse-ws/lecture-examples/src/week11/polymorphism/Cow.java}
\href{https://github.com/bjornregnell/lth-eda016-2015/blob/master/lectures/examples/eclipse-ws/lecture-examples/src/week11/polymorphism/Food.java}{lecture-examples/src/week11/polymorphism/Cow.java}
\end{Slide}


\begin{Slide}{Övning på polymorfism} \footnotesize
\Emph{Övning}: Med papper och penna, implementera klasserna \code{Vegetarian}, \code{Cucumber} och \code{Pig}. Diskutera gärna parvis hur det kommer att bli när man skapar olika slags matobjekt och anropar metoden \code{show()}.
\begin{center}
\tikzstyle{umlclass}=[rectangle, draw=black,  thick, anchor=north, text width=2cm, rectangle split, rectangle split parts = 3]
\begin{tikzpicture}[thick,scale=0.8, every node/.style={scale=0.8}]
\node (Food) [umlclass, rectangle split parts = 2, text width=3cm]  {
            \textit{\textbf{\centerline{Food}}}
            \nodepart[]{second}getWeight(): Int \\ show(): void
        };
        
\node (Vegetarian) [umlclass, rectangle split parts = 2, below = of Food,  xshift=-3cm, text width=3cm]  {
            \textit{\textbf{\centerline{Vegetarian}}}
            \nodepart[]{second} \vspace{1em}
        };  
        
\node (Cucumber) [umlclass, rectangle split parts = 2, below = of Vegetarian, xshift=-2cm]  {
            \textbf{\centerline{Cucumber}}
            \nodepart[]{second}\vspace{1em}
        }; 
        
\node (Tomato) [umlclass, rectangle split parts = 2, below = of Vegetarian, right = of Cucumber, xshift=-0.5cm]  {
            \textbf{\centerline{Tomato}}
            \nodepart[]{second}\vspace{1em}
        };           
                
\node (Animal) [umlclass, rectangle split parts = 2, below = of Food, right = of Vegetarian, xshift=1.5cm, text width=2.5cm,  text width=3cm]  {
            \textit{\textbf{\centerline{Animal}}}
            \nodepart[]{second}\small getSound(): String
        }; 
        
\node (Cow) [umlclass, rectangle split parts = 2, below = of Animal, xshift=-1.5cm]  {
            \textbf{\centerline{Cow}}
            \nodepart[]{second}\vspace{1em}
        }; 
        
\node (Pig) [umlclass, rectangle split parts = 2, below = of Animal, right = of Cow, xshift=-0.5cm]  {
            \textbf{\centerline{Pig}}
            \nodepart[]{second}\vspace{1em}
        };         
                       
\draw[umlarrow] (Animal.north) -- ++(0,0.5) -| (Food.south);    
\draw[umlarrow] (Vegetarian.north) -- ++(0,0.5) -| (Food.south);       
\draw[umlarrow] (Cow.north) -- ++(0,0.5) -| (Animal.south);            
\draw[umlarrow] (Pig.north) -- ++(0,0.5) -| (Animal.south);        
\draw[umlarrow] (Cucumber.north) -- ++(0,0.5) -| (Vegetarian.south);            
\draw[umlarrow] (Tomato.north) -- ++(0,0.5) -| (Vegetarian.south);                    
\end{tikzpicture}
\end{center}
\end{Slide}

\begin{Slide}{Polymorfism med referensvariabler, listor och vektorer}
  \begin{minipage}{0.6\linewidth}   
\footnotesize\href{https://github.com/bjornregnell/lth-eda016-2015/tree/master/lectures/examples/eclipse-ws/lecture-examples/src/week11/polymorfism}{lecture-examples/src/week11/polymorfism}:


\begin{Code}[basicstyle=\ttfamily\fontsize{6}{7}\selectfont, numberstyle=,numbers=left]
// Food f = new Food(42);   // compile error
Food t1 = new Tomato(42);
Tomato t2 = new Tomato(42); 
// t2 = new Cucumber(42);   // compile error
ArrayList<Food> foodList = new ArrayList<Food>();
foodList.add(t1);
foodList.add(t2);
foodList.add(new Pig(84));
foodList.add(new Cow(168));
foodList.add(new Cucumber(21));
for (Food f: foodList){
    f.show();
    int weight = f.getWeight();
    // String sound = f.getSound(); // compile error
    System.out.println("Weight: " + weight);
}
Animal[] animalArray = 
    {new Pig(100), new Cow(500), new Pig(100)};
for (Animal a: animalArray){
    String sound = a.getSound();
    System.out.println(sound);
}
\end{Code}
\Emph{Övning}: Rita minnet efter raderna 5, 10, 18
\end{minipage}
\hspace{0.5cm}
\begin{minipage}[]{0.2\linewidth}  \fontsize{7}{8}\selectfont
\begin{verbatim}
I am abstract Food!
I am abstract Vegetarian!
I am a concrete Tomato!
Weight: 42
I am abstract Food!
I am abstract Vegetarian!
I am a concrete Tomato!
Weight: 42
I am abstract Food!
I am abstract Animal!
I am a concrete Pig!
Weight: 84
I am abstract Food!
I am abstract Animal!
I am a concrete Cow!
Weight: 168
I am abstract Food!
I am abstract Vegetarian!
I am a concrete Cucumber!
Weight: 21
Nöff Nöff!
Muuuu!
Nöff Nöff!
\end{verbatim}
  \end{minipage}
\end{Slide}

\subsection{Definitiva metoder och klasser}
\begin{Slide}{Definitiva metoder med \texttt{final}}
\begin{Code}
public class ChessAlgorithm {
    public static final int WHITE = 1;
    public static final int BLACK = 2;
    //...
    final int getFirstPlayer() {
        return WHITE;
    }
    //...
}
\end{Code}
På grund av \code{final} före metoden får ingen subklass överskugga \code{getFirstPlayer()}\\
\href{https://docs.oracle.com/javase/tutorial/java/IandI/final.html}{Se java tutorial om ''Final classes and methods''}
\end{Slide}

\begin{Slide}{Definitiva klasser med \texttt{final class}}
\begin{Code}
public final class MyInteger {
    private final int value;

    public MyInteger(int value) {
        this.value = value;
    }
    
    public int intValue() {
        return value;
    }
}
\end{Code}
På grund av \code{final} före \code{class} får ingen göra \code{extends} på denna klass och då kan vi vara helt säkra på att ingen subklass ändrar beteendet på \code{MyInteger}. \\
Se exempel här: \href{}{}
\end{Slide}


\begin{Slide}{Regler för grupplabbar}
\begin{itemize}
\item Diskutera i din samarbetsgrupp hur ni vill dela upp ansvaret och arbetet för olika delar av ko
den. Det är lämpligt att varje klass har en huvudansvarig. Flera kan hjälpas åt med samma klass, t.e
x. genom att implementera olika metoder.
\item När ni redovisar er lösning ska ni också kunna redogöra för hur ni delat upp ansvar och arbete
 mellan er. Var och en redovisar sina delar.
\item Grupplaborationer görs i huvudsak som \Alert{hemuppgift}. Salstiden används primärt för redovisning.
\end{itemize}
Diskutera gärna med handledare på resurstid om ni behöver hjälp med hur ni ska dela upp arbetet.
\end{Slide}

\begin{Slide}{Morgondagens föreläsning: Grumligtlådan}
\end{Slide}

\begin{Slide}{Inför nästa vecka: Algoritmer}
\begin{itemize}
\item Repetera algoritmer: min/max, linjärsökning, registrering
\item Nästa vecka: \\ binärsökning, sortering, algoritmisk komplexitet
\end{itemize}
\end{Slide}


\end{document}