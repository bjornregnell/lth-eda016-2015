\documentclass{lecturenotes}

\renewcommand{\vecka}{11}
\newcommand{\tema}{Polymorfism}

\setbeamertemplate{footline}[frame number]
\title[Föreläsningsanteckningar EDA016, 2015]{EDA016 Programmeringsteknik för D}
\subtitle{Läsvecka \vecka: \tema}
\author{Björn Regnell}
\institute{Datavetenskap, LTH}
\date{Lp1-2, HT 2015}

%%%%%%%%%%%%%%%%%%%%%%%%%%%%%%%%%%%%%%

\begin{document}

\frame{\titlepage}
\setnextsection{\vecka}
\section[Vecka \vecka: \tema]{\tema}
\frame{\tableofcontents}

\subsection{Att göra denna vecka}
\begin{Slide}{Att göra i Vecka \vecka: Förstå arv och polymorfism.}
\begin{enumerate}
\item Läs följande kapitel i kursboken:   \\  
Begrepp: polymorfism, klassificering, polymorfa variabler, statisk och dynamisk bindning, överskugga (override), virtuell metod
\item Gör övning 10: arv
\item Träffas i samarbetsgrupper och hjälp varandra 
\item Gör Lab 9: grupplabb TurtleRace
\end{enumerate}
\end{Slide}


\subsection{Repetition: arv}
\begin{Slide}{Repetition: Vad är arv? Motivering och terminologi}\footnotesize
\begin{itemize}
\item Med hjälp av \Emph{arv} mellan klasser kan man göra så att en klass \Emph{ärver} (''får med sig'') innehållet i en \textit{annan} klass.
\item Varför vill man det? 
\begin{enumerate}\footnotesize
\item Dela upp ansvar mellan klasser och bryta ut gemensamma delar så att man slipper duplicerad kod.
\item Skapa en klassificering av objekt utifrån relationen  \Emph{X är en Y}.  \\ Exempel 1: En gurka är en grönsak. En tomat är en grönsak. \\ Exempel 2: En cykel är ett fordon. En bil är ett fordon. 
\end{enumerate}
\item Nyckelordet \Key{extends} används för att ange arv i Java. \\ Exempel:   \code|class TalkingRobot extends Robot|
\item Klassen som ärver (utökar) kallas \Emph{subklass}
\item Klassen som blir utökad kallas \Emph{superklass} (även \textit{basklass})
\item Läs mer om arv \Eng{inheritance} här: \scriptsize \url{https://sv.wikipedia.org/wiki/Arv\_\%28programmering\%29} 
\end{itemize}
\end{Slide}

\begin{Slide}{Klassificering av mat}
\begin{center}

\tikzstyle{umlclass}=[rectangle, draw=black,  thick, anchor=north, text width=2cm, rectangle split, rectangle split parts = 3]
\begin{tikzpicture}[thick,scale=0.8, every node/.style={scale=0.8}]
\node (Food) [umlclass, rectangle split parts = 2, text width=3cm]  {
            \textit{\textbf{\centerline{Food}}}
            \nodepart[]{second}getWeight(): Int \\ show(): void
        };
        
\node (Vegetarian) [umlclass, rectangle split parts = 2, below = of Food,  xshift=-3cm, text width=3cm]  {
            \textit{\textbf{\centerline{Vegetarian}}}
            \nodepart[]{second} \vspace{1em}
        };  
        
\node (Cucumber) [umlclass, rectangle split parts = 2, below = of Vegetarian, xshift=-2cm]  {
            \textbf{\centerline{Cucumber}}
            \nodepart[]{second}\vspace{1em}
        }; 
        
\node (Tomato) [umlclass, rectangle split parts = 2, below = of Vegetarian, right = of Cucumber, xshift=-0.5cm]  {
            \textbf{\centerline{Tomato}}
            \nodepart[]{second}\vspace{1em}
        };           
                
\node (Animal) [umlclass, rectangle split parts = 2, below = of Food, right = of Vegetarian, xshift=1.5cm, text width=2.5cm,  text width=3cm]  {
            \textit{\textbf{\centerline{Animal}}}
            \nodepart[]{second}\small getSound(): String
        }; 
        
\node (Cow) [umlclass, rectangle split parts = 2, below = of Animal, xshift=-1.5cm]  {
            \textbf{\centerline{Cow}}
            \nodepart[]{second}\vspace{1em}
        }; 
        
\node (Pig) [umlclass, rectangle split parts = 2, below = of Animal, right = of Cow, xshift=-0.5cm]  {
            \textbf{\centerline{Pig}}
            \nodepart[]{second}\vspace{1em}
        };         
                       
\draw[umlarrow] (Animal.north) -- ++(0,0.5) -| (Food.south);    
\draw[umlarrow] (Vegetarian.north) -- ++(0,0.5) -| (Food.south);       
\draw[umlarrow] (Cow.north) -- ++(0,0.5) -| (Animal.south);            
\draw[umlarrow] (Pig.north) -- ++(0,0.5) -| (Animal.south);        
\draw[umlarrow] (Cucumber.north) -- ++(0,0.5) -| (Vegetarian.south);            
\draw[umlarrow] (Tomato.north) -- ++(0,0.5) -| (Vegetarian.south);                    
\end{tikzpicture}
\end{center}
\end{Slide}

\subsection{Polymorfism}
\begin{Slide}{Polymorfism: referensvariabler och listor}
  \begin{minipage}{0.6\linewidth}   
\footnotesize\href{https://github.com/bjornregnell/lth-eda016-2015/tree/master/lectures/examples/eclipse-ws/lecture-examples/src/week11/polymorfism}{lecture-examples/src/week11/polymorfism}:

\begin{Code}[basicstyle=\ttfamily\fontsize{6}{7}\selectfont, numberstyle=,numbers=left]
// Food f = new Food(42); // compile error
Food f = new Tomato(42);
ArrayList<Food> foodList = new ArrayList<Food>();
foodList.add(f);
foodList.add(new Pig(84));
foodList.add(new Cow(168));
foodList.add(new Cucumber(21));
for (Food f: foodList){
    f.show();
    int weight = f.getWeight();
    // String sound = f.getSound(); // compile error
    System.out.println("Weight: " + weight);
}
Animal[] animalArray = 
  {new Pig(100), new Cow(500), new Pig(100)};
for (Animal a: animalArray){
    String sound = a.getSound();
    System.out.println(sound);
}
\end{Code}
\end{minipage}
\hspace{0.5cm}
\begin{minipage}[]{0.2\linewidth}   \scriptsize
\begin{verbatim}
I am abstract Food!
I am abstract Vegetarian!
I am a concrete Tomato!
Weight: 42
I am abstract Food!
I am an abstract Animal!
I am a concrete Pig!
Weight: 84
I am abstract Food!
I am an abstract Animal!
I am a concrete Cow!
Weight: 168
I am abstract Food!
I am abstract Vegetarian!
I am a concrete Cucumber!
Weight: 21
Nöff Nöff!
Muuuu!
Nöff Nöff!
\end{verbatim}
  \end{minipage}
\end{Slide}

\end{document}