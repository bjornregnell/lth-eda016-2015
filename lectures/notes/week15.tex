\documentclass{lecturenotes}

\renewcommand{\vecka}{15}
\newcommand{\tema}{Avslutning, Utblick}

\setbeamertemplate{footline}[frame number]
\title[Föreläsningsanteckningar EDA016, 2015]{EDA016 Programmeringsteknik för D}
\subtitle{Läsvecka \vecka: \tema}
\author{Björn Regnell}
\institute{Datavetenskap, LTH}
\date{Lp1-2, HT 2015}

%%%%%%%%%%%%%%%%%%%%%%%%%%%%%%%%%%%%%%

\begin{document}

\frame{\titlepage}
\setnextsection{\vecka}
\section[Vecka \vecka: \tema]{\tema}
\frame{\tableofcontents}

\subsection{Att göra denna vecka}
\begin{Slide}{Att göra i Vecka \vecka: Repetition, uppsamling, tentaplugg.}
\begin{enumerate}
\item Uppsamling: \Alert{gör klart och redovisa kvarvarande labbar och inlämningsuppg. på veckans resurstider}
\item Träffas i samarbetsgruppen \& hjälp varandra att tentaplugga.
\item Ingen labbtid denna vecka - använd resurstider för uppsamling och frågor inför tentapluggande.
\item Kolla på tentapluggtipsen på \href{https://github.com/bjornregnell/lth-eda016-2015/blob/master/lectures/notes/week14.pdf}{förra föreläsningen}.
\end{enumerate}
\end{Slide}

\subsection{Tenta}

\begin{Slide}{Extenta: Fortsättning på Sociala-nätverkstentan}
\end{Slide}

\begin{Slide}{Tenta: anmälning och anonyma tentor}
\begin{enumerate}
\item Du \Alert{måste vara godkänd} på \Alert{alla} labbar + inlämningsuppg. för att få tenta.
\item Läs instruktioner och anmäl dig här:
\item Gruppbonus: Din grupptillhörighet och dess bonus meddelas via mejl. Kontakta mig direkt om något är felaktigt eller problematiskt. 
\end{enumerate}
\end{Slide}

\subsection{Inlämningsuppgift}
\begin{Slide}{Inlämningsuppgift Bank; diskussion + utblick}
\Emph{Exempel på klurighet} \\
Specifikationen tillåter uttag som är större än saldot på kontot, men i det textuella användargränssnittet ska felmeddelande ges vid för stora uttag som inte medges. \\ 
\vspace {1em} Hur löser vi det? \\
\pause Genom att implementera logik som förhindrar för stora uttag.
 \\ \vspace {1em} 
 \pause
\begin{itemize}
\item När man använder ett api behöver man förstå vad api:et omfattar och hur det är tänkt att användas. 
\begin{itemize}
\item Vad behöver jag själv kolla? 
\item Vad kollas av koden jag använder? 
\item Vem har ansvar för vad?
\end{itemize}
\item Mer om design-frågor i OMD-kursen.
\end{itemize}
\end{Slide}

\begin{Slide}{Inlämningsuppgift Draw; diskussion + utblick}\footnotesize
\Emph{Exempel på klurighet} \\
Utvidgning: relativ förflyttning med tangentbordet. \\Men klassen \code{se.lth.cs.pt.shapes.Shape} har inga \code{GetX}, \code{GetY}\\ 
\vspace {1em} Hur löser vi det? 
\pause Två olika lösningar:
\begin{itemize}\footnotesize
\item Kopiera  se.lth.cs.pt.shapes.Shapes till mitt eget paket och lägg till metoderna
\item Eftersom attributen \code{x} och \code{y} är \code{protected} i \code{Shape} så kan vi med en ny klass \code{LocatableShape extends Shape} skapa en ny basklass för våra figurer och där implementera  \code{GetX}, \code{GetY}
\end{itemize}
 \pause \vspace{1em} Viktiga generella observationer:
\begin{itemize}\footnotesize
\item Om du har tillgång till källkoden och får kopiera den kan du bygga vidare på koden (en s.k. \href{https://sv.wikipedia.org/wiki/Fork}{fork}), men vad göra om api:et kommer i ny version? 
\item Genom arv kanske det går att fixa det du saknar, men vad göra om api:et kommer i ny version?
\end{itemize}
\end{Slide}

\subsection{Kursutvärdering}
\begin{Slide}{CEQ}
\begin{enumerate}
\item Om CEQ
\end{enumerate}
\end{Slide}

\begin{Slide}{Kursspecifik utvärdering om specifika kursmoment}
\begin{enumerate}
\item Om kursspecifik utvärdering + länk till google forms
\end{enumerate}
\end{Slide}

\subsection{Grumligtlådan}
\begin{Slide}{Grumligtlådan - avslutande diskussion och frågestund}
Diskussion om några av begreppen ur grumligtlådan: 
\begin{itemize}
\item \Emph{overloading}: flera metoder med samma namn
\item \Emph{override}: ersätta metod i subklass
\item \Emph{iterator}: ett objekt med metoder för att iterera över elementen i en samling
\item ''När du säger \Emph{Java} exakt vad menar du då?''
\begin{enumerate}
\item Ett programspråk
\item En virtuell masking och exekveringsplatform
\item En stor uppsättning standardbibliotek med färdiga klasser
\item En global gemenskap av kompatibla öppenkällkodsbibliotek
\end{enumerate}
\item \Emph{volatile}: speciellt nyckelord för att skydd mot samtidig access av gemensamma variabler i jämlöpande trådar. \footnotesize Java har speciella språkkonstruktioner och bibliotek för att hantera trådar som exekverar samtidigt. Detta är ett avancerat och intressant område som ni läser mer om i realtidsprogrammering.
\end{itemize}
\end{Slide}

\Subsection{Utblick}

\begin{Slide}{Utblick: Vart är Java och JVM-teknologin på väg?}\footnotesize
\begin{itemize}
\item En viktig trend är kombinationen av objekt-orientering och funktionsprogrammering och detta visar sig i \href{http://www.oracle.com/technetwork/java/javase/8-whats-new-2157071.html}{Java 8} med sina anonyma funktioner (s.k. lambda expressions) och nya samlingen java.util.stream.
\item I \href{http://www.javaworld.com/article/2984263/java-platform/oracle-says-java-9-modules-will-be-a-boon-for-developers.html}{Java 9} som resultat av ''\href{http://www.infoworld.com/article/2895889/java/oracle-java-9-future-lego-like.html}{Project Jigsaw}'' kommer mycket handla om modularisering av JDK som möjliggör skräddarsydda, mindre runtime-versioner av JVM.	
\item Mycket av utmaningarna handlar om att dra nytta av många kärnor och parallellprogrammering blir allt hetare. Därför jobbar man mer och mer med oföränderlig data som gör det lättare att dela upp beräkningar på flera kärnor.
\item Det finns flera livaktiga språk som kör på JVM, t.ex.  \href{https://sv.wikipedia.org/wiki/Scala\_\%28programspr\%C3\%A5k\%29}{Scala} och  \href{https://sv.wikipedia.org/wiki/Clojure}{Clojure} men också \href{https://sv.wikipedia.org/wiki/Javascript}{Javascript} med skriptmotorn \href{https://en.wikipedia.org/wiki/Nashorn\_\%28JavaScript_engine\%29}{Nashorn} 
\item Med kompilatorn för Scala till Javascript \href{http://www.scala-js.org/}{Scala.js} kan man dela kod mellan en server som kör på JVM och en klient som kör i webbläsaren.
\end{itemize}
\end{Slide}

\begin{Slide}{Utblick: Vart är software engineering  på väg?}
Några faktorer som skapar förändringstryck för ökad produktivitet i utvecklingsprocessen:
\begin{itemize}
\item Snabbast vinner. 
\item Jakt på kompetens. 
\item Sakernas internet.
\end{itemize}
\vspace{1em}
Några starka trender:
\begin{itemize}
\item \href{https://en.wikipedia.org/wiki/Open_source}{Open source}
\item \href{https://en.wikipedia.org/wiki/Big_data}{Big Data}
\item \href{https://en.wikipedia.org/wiki/DevOps}{Devops}
\end{itemize}
\end{Slide}
\Subsection{Frågestund}

\Subsection{Om nästa läsperiod: Gäster Anna Axelsson \& Martin Höst}

\begin{Slide}{Ett stort TACK för...}
\begin{itemize}
\item ... att ni kämpat så glatt!
\item ... att ni ställt massor med frågor!
\item ... att det har varit så hög närvaro på föreläsningarna!
\item ... att ni är så konstruktiva och verkligen vill lära er!
\end{itemize}
\vspace{2em} \pause

\Alert{Ett stort LYCKA TILL på vägen till att bli en \\ kompetent och innovativ systemutvecklare!}
\end{Slide}
\end{document}