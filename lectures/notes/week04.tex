\documentclass{lecturenotes}

\renewcommand{\vecka}{4}
\newcommand{\tema}{Aritmetik, Logik \& Datastrukturer}

\setbeamertemplate{footline}[frame number]
\title[Föreläsningsanteckningar EDA016, 2015]{EDA016 Programmeringsteknik för D}
\subtitle{Läsvecka \vecka: \tema}
\author{Björn Regnell}
\institute{Datavetenskap, LTH}
\date{Lp1-2, HT 2015}
 
\begin{document}

\frame{\titlepage}
\setnextsection{\vecka}
\section[Vecka \vecka: \tema]{\tema}
\frame{\tableofcontents}

%%%%%%%%%%%%%%%%%%%%%%%%%%%%%%%%%%%%%%

\subsection{Att göra denna vecka}
\frame{\frametitle{Att göra i Vecka \vecka: Förstå aritmetiska och logiska uttryck, använda klasser mha klass-specifikationer}
\begin{enumerate}
\item Läs följande kapitel i kursboken:\\ 6.1--6.4, 6.8--6.9,  7.1--7.7, 7.10--7.11, 7.13\\ 
Begrepp:heltalsdivition med rest,  typkonvertering, De Morgans lagar, oföränderlighet
\item Gör övning 4: Aritmetik, Logik
\item Träffas i samarbetsgrupper och hjälp varandra förstå 
\item Gör Lab 3: använda färdigskrivna klasser, kvadrat
\end{enumerate}
}


\subsection{Numeriska typer}
\begin{Slide}{Primitiva datatyper i Java}
\begin{center}
\begin{tabular}{lp{6cm}}
Typ & Betydelse \\ \midrule
\Key{byte}, \Key{short}, \Key{int}, \Key{long} & heltal \\
\Key{float}, \Key{double} & reellt tal (tal med decimaldel) \\
\Key{boolean} & logiskt värde (sant eller falskt) \\
\Key{char} & tecken, till exempel bokstav, siffra, specialtecken \\
\end{tabular} 
\end{center}
\end{Slide}

\begin{Slide}{Numeriska typernas storlek samt min- och max-värden}  
\begin{center}
\begin{tabular}{cccc}
Typ&	Bitar&	Min&	Max\\
\hline
\Key{byte}	&$8$	&$-128$&	$+127$\\
\Key{short}	&$16$	&$-32~768$	&$+32~767$\\
\Key{char}	&$16$	&$0$&	$65~535$\\
\Key{int}	&$32$	&$-2~147~483~648$&	$+2~147~483~647$\\
\Key{long}	&$64$	&$\approx -9\cdot 10^{18}$	&$\approx +9\cdot 10^{18}$\\
\Key{float}	&$32$	&$\approx -3.4\cdot 10^{38}$	&$\approx +3.4\cdot 10^{38}$\\
\Key{double}	&$64$	&$\approx -1.8\cdot 10^{308}$	&$\approx +-1.8\cdot 10^{308}$\\
\end{tabular}
\end{center}
För detaljer se \href{https://docs.oracle.com/javase/specs/jls/se8/html/jls-4.html#jls-4.2.3}{Javas språkspecifikation} och\\ \href{https://en.wikipedia.org/wiki/Double-precision_floating-point_format}{IEEE-standarden 754}
\end{Slide}

\begin{Slide}{Implicita  \href{http://docs.oracle.com/javase/tutorial/java/nutsandbolts/datatypes.html}{startvärden} för attribut}
\end{Slide}

\begin{Slide}{Implicit och explicit konvertering mellan numeriska värden}
\end{Slide}


\begin{Slide}{Några aritmetriska uttryck}
\end{Slide}

\begin{Slide}{Var n-te gång, jämt delbart med n}
\end{Slide}

\begin{Slide}{Negering av logiska uttryck med De Morgans lagar}
\end{Slide}


\begin{Slide}{Specifikation av \texttt{SimpleWindow}}
Så här ser delar av specifikationen för \texttt{SimpleWindow} ut i ankboken Appendix C, sidan 309-312.
\begin{ClassSpec}{SimpleWindow}
/** Creates a window and makes it visible. */
SimpleWindow(int width, int height, java.lang.String title);

/** Moves the pen to a new position. */
void moveTo(int x, int y)

/** Moves the pen to a new position while drawing a line. */
void lineTo(int x, int y)
\end{ClassSpec}
Specifikationerna i ankboken liknar (en enklare variant av) javadoc som genereras ur dokumentationskommentarer. Jämför med \href{http://fileadmin.cs.lth.se/cs/Education/EDA016/2015/doc/se/lth/cs/pt/window/SimpleWindow.html}{javadoc för \texttt{SimpleWindow}}
\end{Slide}


\begin{Slide}{Specifikation av klassen \texttt{Square}}
\begin{ClassSpec}{Square}
/** Skapar en kvadrat med övre vänstra hörnet i x,y och med sidlängden side  */
Square(int x, int y, int side);

/** Ritar kvadraten i fönstret w */
void draw(SimpleWindow w);

/** Flyttar kvadraten avståndet dx i x-led, dy i y-led */
void move(int dx, int dy);

/** Tar reda på x-koordinaten för kvadratens läge */
int getX();

/** Tar reda på y-koordinaten för kvadratens läge */
int getY();

/** Tar reda på kvadratens area */
int getArea();
\end{ClassSpec}
\end{Slide}

\begin{Slide}{Implementation av delar av Square-klassen}
\lstinputlisting[language=Java, basicstyle=\ttfamily\tiny\selectfont, numberstyle=, numbers=left,]{../examples/eclipse-ws/lecture-examples/src/week04/Square.java}
\end{Slide}



\Subsection{Oföränderlighet (immutability)}
\begin{Slide}{Förhindra att variabler \href{https://docs.oracle.com/javase/tutorial/essential/concurrency/immutable.html}{ändras} med \texttt{\textbf{final}}}
Attributet \texttt{latinsktNamn} nedan är en \Emph{konstant}.\\ Kompilatorn hjälper oss att kolla så att vi inte råkar ändra på det vi har deklarerat som \Key{final}.
\lstinputlisting[language=Java, basicstyle=\ttfamily\tiny\selectfont, numberstyle=, numbers=left,]{../examples/terminal/final/Constant.java}
\end{Slide}

\begin{Slide}{Oföränderligt objekt}
\lstinputlisting[language=Java, basicstyle=\ttfamily\tiny\selectfont, numberstyle=, numbers=left,]{../examples/terminal/final/ImmutableObject.java}
\end{Slide}

\end{document}