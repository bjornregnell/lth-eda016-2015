\documentclass{lecturenotes}

\renewcommand{\vecka}{14}
\newcommand{\tema}{Extenta}

\setbeamertemplate{footline}[frame number]
\title[Föreläsningsanteckningar EDA016, 2015]{EDA016 Programmeringsteknik för D}
\subtitle{Läsvecka \vecka: \tema}
\author{Björn Regnell}
\institute{Datavetenskap, LTH}
\date{Lp1-2, HT 2015}

%%%%%%%%%%%%%%%%%%%%%%%%%%%%%%%%%%%%%%

\begin{document}

\frame{\titlepage}
\setnextsection{\vecka}
\section[Vecka \vecka: \tema]{\tema}
\frame{\tableofcontents}

\subsection{Att göra denna vecka}
\begin{Slide}{Att göra i Vecka \vecka: Inlämningsuppgift + repetition.}
\begin{enumerate}
\item Gör extraövningar och studera sedan lösningsförslagen \\ {\scriptsize \url{http://fileadmin.cs.lth.se/cs/Education/EDA016/exercises/extraexercises.pdf}}
\item Träffas i samarbetsgruppen \& hjälp varandra att repetera
\item Jobba med inlämningsuppgift: \\ utveckla koden, förbered redovisningen, redovisa
\item Läs igenom följande extenta: \href{http://fileadmin.cs.lth.se/cs//Education/grundkurs/extentor/150318.pdf}{150318.pdf (socialt nätverk)}
\end{enumerate}
\end{Slide}

\subsection{Tentatips}
\begin{Slide}{Före tentan: tentapluggningstips}\footnotesize
\begin{enumerate}
\item Gör \emph{alla} övningar i kompendiet på papper \emph{innan} du ger dig på extentorna. 
\item Läs igenom \href{https://github.com/bjornregnell/lth-eda016-2015/tree/master/lectures/notes}{föreläsningsanteckningar} och \href{https://github.com/bjornregnell/lth-eda016-2015/tree/master/lectures/examples/eclipse-ws/lecture-examples/src}{exempelkod}.
\item Läs boken enligt  \href{http://cs.lth.se/eda016/foerelaesningar-2015/}{läsanvisningarna} på kurshemsidan.
\item Studera \Emph{java snabbref} \Alert{mycket noga} så att du vet vad som är givet och var det står, så att du kan hitta det du behöver snabbt.
\item Gör minst en extenta som om det vore \Alert{skarpt läge}: 
\begin{enumerate}\footnotesize
\item Avsätt 5 ostörda timmar (stäng av telefon, dator etc).
\item Inga hjälpmedel. Bara java snabbref.
\item Förbered dryck och tilltugg.
\end{enumerate}
\item Skapa och \Emph{memorera} en personlig \Emph{checklista} med fel du brukar göra när du programmerar som även inkluderar småfel som en utvecklingsmiljö (t.ex. Eclipse) hittar.
\item Tänk igenom hur du ska disponera dina 5 timmar på tentan.
\item Läs igenom 3-4 extentor för att bilda dig en uppfattning om variationen.
\end{enumerate}
\end{Slide}

\begin{Slide}{På tentan: tips och rekommendationer} \footnotesize
\begin{enumerate}
\item Läs igenom \Alert{hela} tentan först. \\ \Emph{Varför?} Förstå helheten. Delarna hänger ihop.
\item Notera och begrunda domänspecifika begrepp och definitioner. \\ \Emph{Varför?} Begreppen är avgörande för förståelsen av uppgiften.
\item Notera förenklingar, antaganden och specialfall. \\ \Emph{Varför?} Uppgiften blir väsentligt enklare om du inte behöver hantera vissa specialfall etc.
\item \Alert{Fråga} tentamensansvarig om du inte förstår uppgiften -- speciellt om det finns misstänkta felaktigheter eller förmodat oavsiktliga oklarheter. \\ \Emph{Varför?} Det är inte lätt att konstruera en ''perfekt'' tenta. Du får fråga vad du vill men det är inte säkert du får svar :)
\item Läs specifikationskommentarerna och metodsignaturerna i alla givna klass-specifikationer \Alert{mycket noga}. \\ \Emph{Varför?} Det är ett vanligt misstag att förbise de ledtrådar som ges där.
\item Återskapa din memorerade personliga checklista för vanliga fel som du brukar göra och avsätt tid till att gå igenom den.
\item Lämna in ett försök även om du vet att lösningen inte är fullständig. 
\end{enumerate}
\end{Slide}

\subsection{Grumligtlådan}
\begin{Slide}{Grumligtlådan}\footnotesize
Många ämnen i senaste grumligtlådan illustreras under tentagenomgången.\\ \vspace{1em}
\begin{tabular}{r|l}
\#Lappar  & Ämne                         \\ \hline
6  & StringBuilder\\
3  & Vektorer, ArrayList\\
2  & Implementering och användning av klasser\\
2  & Sorteringsalgoritmer\\
2  & Static\\
1 & Arv\\
1  & Generics\\
1  & for-each-sats\\
1  & Flera metoder med samma namn\\
1  & Matris\\
1  & När du säger "Java" exakt vad menar du då?\\
1  & Iterator\\
1 & Volatile Image\\
\end{tabular}
\end{Slide}

\subsection{Lösning av extenta}
\begin{Slide}{Genomgång av extenta}
Vi går igenom en extenta på tavlan och diskuterar vanliga fel, rättningsmall etc.\\ \vspace{2em}
Mars 2015, tema ''socialt nätverk''
\begin{enumerate}
\item Tenta: \href{http://fileadmin.cs.lth.se/cs//Education/grundkurs/extentor/150318.pdf}{150318.pdf}
\item Lösning: \href{http://fileadmin.cs.lth.se/cs//Education/grundkurs/extentor/sol-150318.pdf}{sol-150318.pdf}
\end{enumerate}
\vspace{1em}
''Tentan handlar om att skriva klasser som ska ingå i ett program för att bygga sociala nätverk (liknande
Facebook, LinkedIn eller Instagram).''

\end{Slide}

\begin{Slide}{Några vanliga fel på denna tenta}\footnotesize
\begin{enumerate}
\item Inser inte att man kan komma åt attributet \code{f.userId} i \code{findCommonFirends}. Privat synlighet gäller alltså på klassnivå inte instansnivå.
\item Missat att göra \code{owner.addActivity(this)} i konstruktorn till \code{Activity}
\item Missat att använda befintlig metod. Exempel: det finns en metod \code{findFriend} men man skriver ändå kod för att söka upp objektet varje gång man behöver det.
\item Många brukar missar på registreringsuppgifter (uppg.3 i denna tenta) för att man inte förstått lösningsidén med hur indexet i en registreringsvektor används.
\end{enumerate}
\end{Slide}

\end{document}